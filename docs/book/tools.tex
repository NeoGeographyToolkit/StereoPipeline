\chapter{Tools}

This chapter gives a quick overview of various tools that are provided by
Ames Stereo Pipeline and a summary of their command line options.

%--------------------------------------------------------------------------
%                                ASP DEBUGGING
%--------------------------------------------------------------------------

\section{stereo}
\label{stereo}

The \texttt{stereo} program is the primary workhorse of the Ames
Stereo Pipeline.  It takes a stereo pair of images that overlap and
creates an output point cloud image that can be processed into a 3D
model or DEM using the \texttt{point2mesh} or \texttt{point2dem}
programs, respectively.  

\medskip

Usage:
\begin{verbatim}
    stereo [options] <Left_input_image> <Right_input_image> <output_file_prefix>
\end{verbatim}

\medskip

This release of the stereo pipeline been specifically designed to
process USGS ISIS \verb=.cub= files.  However, the stereo pipeline
does have the capability to process other types of stereo image pairs
(e.g. image files with a CAHVOR camera model from the NASA MER
rovers).  If you would like to experiment with these features, please
contact us for more information.

The \verb=<output_file_prefix>= is prepended to all output data files.
For example, setting \verb=<output_file_prefix>= to `\verb=out=' will
yield files with names like \verb=out-L.tif= and \verb=out-PC.tif=.
To keep stereo pipeline results organized in sub-directories, we
recommend using an output prefix like `\verb=results-10-12-09/out='
for \verb=<output_file_prefix>=.  The \verb=stereo= program will
create a directory called \verb=results-10-12-09/= and place files
named \verb=out-L.tif=, \verb=out-PC.tif=, etc. in that directory.

\begin{longtable}{|l|p{10cm}|}
\caption{Command-line options for stereo}
\label{tbl:stereo}
\endfirsthead
\endhead
\endfoot
\endlastfoot
\hline
Option & Description \\ \hline \hline
\verb#--help# & Display this table\\ \hline
\verb#--cache arg (=1800)# & Set the maximium cache available in megabytes\\ \hline
\verb#--threads arg (=0)# & Set the number threads to use. 0 means use default defined .vwrc\\ \hline
\verb#--session-type arg# & Select the stereo session type to use for processing. Usually the program can select this automatically for file extension. [options pinhole isis]\\ \hline
\verb#--stereo-file arg (=./stereo.default)# & Define the stereo.default file to use\\ \hline
\verb#--entry-point# & Pipeline entry point [options 1-4]\\ \hline
\verb#--debug-level# & Sets the output debugging level\\ \hline
\verb#--draft-mode arg# & Cause the pyramid correlator to save out debug imagery named with this prefix\\ \hline
\verb#--optimized-correlator# & Cause scale space search to not be performed\\ \hline
\end{longtable}

\subsection{Entry Points}
\label{entrypoints}

The {\tt stereo -e <number>} option can be used to restart a {\tt
  stereo} job partway through the stereo correlation process.
Restarting can be handy when debugging while iterating on {\tt
  stereo.default} settings.

Stage 0 (Preprocessing) normalizes the two images and aligns them
(thus non-projected images are easier to work with) by locating
interest points and matching them in both images. The program is
designed to reject outlying interest points.  This stage writes out
the pre-aligned images and the image masks.

Stage 1 (Disparity Map Initialization) performs pyramid correlation and builds a rought disparity map that is used to seed the sub-pixel refinement phase.

Stage 2 (Sub-pixel Refinement) performs sub-pixel correlation that
refines the disparity map.

Stage 3 (Outlier Rejection and Hole Filling) performs filtering of the
disparity map and (optionally) fills in holse using an inpainting
algorithm.  This phase also creates a ``good pixel'' map.

Stage 4 (Triangulation) generates a 3D point cloud from the disparity
map.

\section{disparitydebug}
\label{disparitydebug}

Produces output images for debugging disparity images created from
\verb#stereo#. The {\tt stereo} tool produces several different
versions of the disparity map; the most important ending with
extentions \verb#*-D.exr# and \verb#*-F.exr#. (see Appendix
\ref{chapter:outputfiles} for more information.)  These raw disparity
map files can be useful for debugging because they contain raw
disparity values as measured by the correlator; however they cannot be
directly visualized or opened in a conventional image browser.  The
\verb#disparitydebug# tool converts a single disparity map file into
two normalized TIFF image files (\verb#*-H.tif# and \verb#*-V.tif#,
conaining the horizontal and vertical components of disparity,
respectively) that can be viewed using any image display program.

{\tt disparitydebug} will also print out the range of disparity values
in a disparity map, that can serve as useful summary statistics when
tuning up the up the search range settings in the {\tt stereo.default}
file.

\begin{longtable}{|l|p{10cm}|}
\caption{Command-line options for disparitydebug}
\label{tbl:disparitydebug}
\endfirsthead
\endhead
\endfoot
\endlastfoot
\hline
Options & Description \\ \hline \hline
\verb#--help# & Display this table \\ \hline
\verb#--cache arg (=1024)# & Cache size, in megabytes \\ \hline
\verb#--input-file arg# & Explicitly specify the input file \\ \hline
\verb#-o arg# & specify the output file prefix \\ \hline
\verb#-t arg (=tif)# & Specify the outfile type \\ \hline
\verb#-d arg (=29)# & Set the debugging output level \\ \hline
\verb#--float-pixels# & Save the resulting debug images as 32 bit floating point files ( if supported by the slected file type ) \\ \hline
\end{longtable}

%--------------------------------------------------------------------------
%                           VISUALIZATION TOOLS
%--------------------------------------------------------------------------

\section{point2dem}
\label{point2dem}

Produces a GeoTIFF terrain model or an orthographic image from a point cloud image produced by the {\tt stereo} command.

Example:
\begin{verbatim}
point2dem <output-prefix>-PC.tif -o stereo/filename --xyz -r moon --default-value -10000 -n
\end{verbatim}

This produces a digital elevation model that has been referenced to
the lunar spheroid of 1737.4-km.  Pixels with nodata will be set to a
value of -10000, and the rusulting DEM will be saved in a simple
cylindrical map projection.  The resulting DEM is stored by default as
a one channel, 32-bit floating point GeoTIFF file.

The {\tt -n} option creates an 8-bit, normalized version of the DEM
that can be easily loaded into a standard image viewing application
for debugging.

Example:
\begin{verbatim}
point2dem <output-prefix>-PC.tif -o stereo/filename --xyz -r moon
 --orthoimage <output-prefix>-L.tif
\end{verbatim}

This command takes the left input image and orthographically projects
it onto the 3D terrain produced by the Stereo Pipeline.  The resulting
{\tt *-DRG.tif} file will be saved as an 8-bit GeoTIFF image in a
simple cylindrical map projection.

\begin{longtable}{|l|p{10cm}|}
\caption{Command-line options for point2dem}
\label{tbl:point2dem}
\endfirsthead
\endhead
\endfoot
\endlastfoot
\hline
Options & Description \\ \hline \hline
\verb#--help# & Display this table \\ \hline
\verb#--default-value# & Explicitly set the default missing pixel value. By default, the min z value is used. \\ \hline
\verb#--use-alpha# & Create images that have an alpha channel \\ \hline
\verb#-s arg(=0)# & Set the DEM post size (if this value is 0, the post spacing size is computed for you) \\ \hline
\verb#-n# & Also write a normalized version of the DEM (for debugging) \\ \hline
\verb#--orthoimage arg# & Write an orthoimage based on the texture file given as an argument to this command line option \\ \hline
\verb#--grayscale# & Use grayscale image processing for creating the orthoimage \\ \hline
\verb#--offset-files# & Also write a pair of ascii offset files (for debugging) \\ \hline
\verb#--cache arg (=2048)# & Cache size, in megabytes \\ \hline
\verb#--input-file arg# & Explicitly specify the input file \\ \hline
\verb#--texture-file arg# & Explicitly specify the texture file \\ \hline
\verb#-o arg# & Specify the output prefix \\ \hline
\verb#-t arg (=tif)# & Specify the output file type \\ \hline
\verb#-d arg (=29)# & Set the debugging output level. (0-50+) \\ \hline
\verb#--xyz-to-lonlat# & Convert from XYZ coordinates to LLA coordinates \\ \hline
\verb#-r arg# & Set a reference surface to a hard coded value (one of [mmon, mars]). This will override manually set datum information. \\ \hline
\verb#--semi-major-axis arg (=0)# & Set the dimensions of the datum \\ \hline
\verb#--semi-minor-axis arg (=0)# & Set the dimensions of the datum \\ \hline
\verb#--x-offset arg (=0)# & Add a horizontal offset to the DEM \\ \hline
\verb#--y-offset arg (=0)# & Add a horizontal offset to the DEM \\ \hline
\verb#--z-offset arg (=0)# & Add a vertical offset to the DEM \\ \hline
\verb#--sinusoidal# & Save using a sinusoidal projection \\ \hline
\verb#--mercator# & Save using a Mercator projection \\ \hline
\verb#--transverse-mercator# & Save using transverse Mercator projection \\ \hline
\verb#--orthographic# & Save using an orthographic projection \\ \hline
\verb#--stereographic# & Save using a stereographic projection \\ \hline
\verb#--lambert-azimuthal# & Save using a Lambert azimuthal projection \\ \hline
\verb#--utm arg# & Save using a UTM projection with the given zone \\ \hline
\verb#--proj-lat arg# & The center of projection latitude (if applicable) \\ \hline
\verb#--proj-lon arg# & The center of projection longitude (if applicable) \\ \hline
\verb#--proj-scale arg# & The projection scale (if applicable) \\ \hline
\verb#--rotation-order arg (=xyz)# & Set the order of an euler angle rotation applied to the 3D points prior to DEM rasterization \\ \hline
\verb#--phi-rotation arg (=0)# & Set a rotation angle phi \\ \hline
\verb#--omega-rotation arg (=0)# & Set a rotation angle omega \\ \hline
\verb#--kappa-rotation arg (=0)# & Set a rotation angle kappa \\ \hline
\end{longtable}

\section{point2mesh}
\label{point2mesh}

Produces a mesh surface that can be visualized in {\tt osgviewer},
which is a standard 3D viewing application that is part of the open
source OpenSceneGraph package.  \footnote{OpenSceneGraph is not bundled
with the Stereo Pipeline.  You must download and install this package
seperately from {\tt http://www.openscenegraph.org/.}}

Unlike DEMs, The 3D mesh is not meant to be used as a finished
scientific product.  Rather, it cn be used for fast visualization or
to create a cool 3D view of the data generated. 

\verb#point2mesh# requires a point cloud file and optionally the left
texture file ({\tt <output-prefix-PC.tif} and
{\tt <output-prefix>-L.tif}). When a texture file is not provided, a
1D texture is applied in the local Z direction that produces a rough
rendition of a contour map.  In either case, \verb#point2mesh# will
produce a \verb#<output-prefix>.ive# file that contains the 3D model
in OpenSceneGraph format.

Two options bear pointing out: the \verb#-l# flag indicates that
synthetic lighting should be activated for the model, which can make
is easier to see fine detail in the model by providing some real-time,
interactive hillshading.  The \verb#-s# flag sets the sub-sampling
rate, and dictates the degree to which the 3D model should be
simplified.  (For 3D reconstructions, this can be essential for
producing a model that can fit in memory.)  The default value is 10,
meaning every 10th point is used in the X and Y directions. In other
words that mean only $1/10^2$ of the points are being used to create
the model. Adjust this sampling rate according to how much detail is
desired, but remember that large models will hurt the framerate of the
3D viewer..

Example:
\begin{verbatim}
      point2mesh -l -s 2 <output-prefix>-PC.tif <output-prefix>-L.tif
\end{verbatim}

To view the resulting \verb#*.ive#, use \verb#osgviewer#

\begin{verbatim}
      # Fullscreen
      osgviewer output.ive
      # or Windowed
      osgviewer output.ive --window 50 50 1000 1000
\end{verbatim}

Inside \verb#osgviewer#, the keys L, T, and W can be used to toggle on
and off lighting, texture, and wireframe modes.  The left, middle, and
right mouse buttons control rotation, panning, and zooming of the
model.

\begin{longtable}{|l|p{10cm}|}
\caption{Command-line options for point2mesh}
\label{tbl:point2mesh}
\endfirsthead
\endhead
\endfoot
\endlastfoot
\hline
Options & Description \\ \hline \hline
\verb#--help# & Display this table \\ \hline
\verb#--simplify-mesh arg# & Run OSG Simplifier on mesh, 1.0 = 100\% \\ \hline
\verb#--smooth-mesh# & Run OSG Smoother on mesh \\ \hline
\verb#--use-delaunay# & Uses the delaunay triangulator to create a surface from the point cloud. This is not recommended for point clouds with noise issues. \\ \hline
\verb#-s arg (=10)# & Sampling step size for mesher. \\ \hline
\verb#--input-file arg# & Explicitly specify the input file \\ \hline
\verb#--texture-file arg# & Explicitly specify the texture file \\ \hline
\verb#-o arg# & Specify the output prefix \\ \hline
\verb#-t arg (=ive)# & Specify the output file type \\ \hline
\verb#-l# & Enables shades and light on the mesh \\ \hline
\verb#--center# & Center the model around the origin. Use this option if you are experiencing numerical precision issues. \\ \hline
\verb#--rotation-order arg (=xyz)# & Set the order of an euler angle rotation applied to the 3D points prior to DEM rasterization \\ \hline
\verb#--phi-rotation arg (=0)# & Set a rotation angle phi \\ \hline
\verb#--omega-rotation arg (=0)# & Set a rotation angle omega \\ \hline
\verb#--kappa-rotation arg (=0)# & Set a rotation angle kappa \\ \hline
\end{longtable}

%% \section{orthoproject}
%% \label{orthoproject}

%% Map projects imagery on to point clouds.

%% Example:
%% \begin{verbatim}
%% orthoproject -t isis filename-DEM.tif filename.cub filename.isis_adjust \\
%%         filename-DRG.tif --nodata -10000 --ppd 256
%% \end{verbatim}

%% \begin{longtable}{|l|p{10cm}|}
%% \caption{Command-line options for orthoproject}
%% \label{tbl:orthoproject}
%% \endfirsthead
%% \endhead
%% \endfoot
%% \endlastfoot
%% \hline
%% Options & Description \\ \hline \hline
%% \verb#--help# & Display this table \\ \hline
%% \verb#--mpp arg# & Specify the output resolution of the orthoimage in meters per pixel \\ \hline
%% \verb#--ppd arg# & Specify the output resolution of the orthoimage in pixels per degree \\ \hline
%% \verb#--nodata-value arg# & Specify the pixel used in this DEM to denote missing data \\ \hline
%% \verb#-m# & Match the georeferencing parameters and dimensions of the input DEM \\ \hline
%% \verb#--min arg# & Explicitly specify the range of the normalization (for ISIS images only) \\ \hline
%% \verb#--max arg# & Explicitly specify the range of the normalization (for ISIS images only) \\ \hline
%% \verb#--cache arg# & Cache size, in megabytes \\ \hline
%% \verb#-t arg (=pinhole)# & Select the stereo session type to use for processing. [default: pinhole] \\ \hline
%% \verb#-d arg (=29)# & Set the debugging output level. (0-50+) \\ \hline
%% \end{longtable}

\section{orbitviz}
\label{orbitviz}

Produces a Google Earth kml file useful for visualizating camera
position. The input for this tool is one or more \verb#*.cub# files.

\begin{figure}[ht]
  \begin{center}
  \includegraphics[width=6in]{images/orbitviz_ge_result.png}
  \end{center}
  \caption{ Example of a KML visualization produced with {\tt
      orbitviz} depicting camera locations for the Apollo 15 Metric
    Camera during orbit 33 of the Apollo command module.}
  \label{fig:orbitviz_example}
\end{figure}

\begin{longtable}{|l|p{10cm}|}
\caption{Command-line options for orbitviz}
\label{tbl:orbitviz}
\endfirsthead
\endhead
\endfoot
\endlastfoot
\hline
Options & Description \\ \hline \hline
\verb#--help# & Display this table \\ \hline
\verb#-o arg (=orbit.kml)# & Specifies the output file name \\ \hline
\verb#-s arg (=1)# & Scale the size of the coordinate axes by this amount. Ex: To scale axis sizes up to earth size, use 3.66 \\ \hline
\verb#--use-simple-placemarks# & Draw simple icons at camera locations, instead of a coordinate model \\ \hline
\end{longtable}


%--------------------------------------------------------------------------
%                       BUNDLE ADJUSTMENT TOOLS
%--------------------------------------------------------------------------

\section{isis\_adjust}

Bundle Adjustment for ISIS images. This tool supports adjustment of
linescan cameras as well as simple frame cameras. For an in depth view
into how to use this tool, please read Chapter
\ref{ch:bundle_adjustment}.

\begin{longtable}{|l|p{10cm}|}
\caption{Command-line options for isis\_adjust}
\label{tbl:isise_adjust}
\endfirsthead
\endhead
\endfoot
\endlastfoot
\hline
Options & Description \\ \hline \hline
\verb#--help# & Display this table \\ \hline
\verb#-c arg# & Load a control network from a file \\ \hline
\verb#--cost-function arg (=L2)# & Choose a robust cost function from [PseudoHuber, Huber, L1, L2, Cauchy] \\ \hline
\verb#--bundle-adjuster arg (=Sparse)# & Choose a bundle adjustment version from [Ref, Sparse, RobustRef, RobustSparse] \\ \hline
\verb#--disable-camera-const# & Disable the camera constraint error. This allows the cameras to move to pretty much anywhere. \\ \hline
\verb#--disable-gcp-const# & Disable the GCP constraint error. \\ \hline
\verb#--gcp-scalar arg (=1)# & Sets a scalar to multiply to the sigmas (uncertainty) defined for the gcps. GCP sigmas are defined in the .gcp files. \\ \hline
\verb#-l arg# & Set the initial value of the LM paramter g\_lambda. If not set the algorithm will find the optimium starting point. \\ \hline
\verb#--min-matches arg (=30)# & Set the minimum number of matches between images that will be considered. \\ \hline
\verb#--max-iterations arg (=25)# & Set the maximum number of iterations \\ \hline
\verb#--poly-order arg (=0)# & Set the order of the polynomial used adjust the camera properties. If using a frame camera, leave at 0 (meaning scalar offsets). For line scan cameras try 2. \\ \hline
\verb#--position-sigma arg (=100)# & Set the sigma (uncertainty) of the spacecraft position. (meters) \\ \hline
\verb#--pose-sigma arg (=.1)# & Set the sigma (uncertainty) of the spacecraft pose. (radians) \\ \hline
\verb#-r arg (=10)# & Changes the detail of the Bundle Adjustment Report. Valid values are 0 to 100 \\ \hline
\verb#--robust-threshold arg (=10)# & Set the threshold for robust cost functions. \\ \hline
\verb#-s# & Saves all camera/point/pixel information between iterations for later viewing in Bundlevis \\ \hline
\verb#--seed-with-previous# & Use previous isis\_adjust files at starting for this run \\ \hline
\verb#--write-isis-cnet-also# & Writes an ISIS style control network \\ \hline
\verb#--write-kml arg# & Selecting this will cause a kml to be written with the GCPs. Set this flag with 1 and it will also write all the 3D position estimates of the points it is tracking in the KML. \\ \hline
\end{longtable}

%% \section{bundle\_adjust}
%% \label{bundle_adjust}

%% A generic bundle adjustment tool for ISIS images.  See Chapter
%% \ref{ch:bundle_adjustment} for more information.

%% \begin{longtable}{|l|p{10cm}|}
%% \caption{Command-line options for bundle\_adjust}
%% \label{tbl:bundle_adjust}
%% \endfirsthead
%% \endhead
%% \endfoot
%% \endlastfoot
%% \hline
%% Options & Description \\ \hline \hline
%% \verb#--help# & Display this table \\ \hline
%% \verb#-t arg(=isis)# & Select the stereo session type to use for processing. \\ \hline
%% \verb#-c arg# & Load a control network from a file. \\ \hline
%% \verb#-l arg# & Set the initial value of the LM parameter lambda \\ \hline
%% \verb#--robust-threshold arg (=10)# & Set the threshold for robust cost functions \\ \hline
%% \verb#-s# & Savae all camera information between iterations to iterCameraParam.txt, it also saves point locations for all iterations in iterPointsParam.txt \\ \hline
%% \verb# --min-matches arg (=30)# & When building a new control network, sets the minimum number of matches in a match to be added to the control network at a time. \\ \hline
%% \verb# -r arg (=10)# & Changes the detail of the Bundle Adjustment Report ( values range from 0 to 100 ). \\ \hline
%% \end{longtable}

\section{bundlevis}
\label{bundlevis}

Bundle Adjustment result visualizer.  See Chapter
\ref{ch:bundle_adjustment} for more information.

\begin{longtable}{|l|p{10cm}|}
\caption{Command-line options for bundlevis}
\label{tbl:bundlevis}
\endfirsthead
\endhead
\endfoot
\endlastfoot
\hline
Options & Description \\ \hline \hline
\verb#--help# & Display this table \\ \hline
\verb#-c arg# & Load the camera parameters for each iteration from this file \\ \hline
\verb#-p arg# & Load the 3D points parameters for each iteration from this file \\ \hline
\verb#-x arg# & Load pixel information data. Allowing for an illustration of the pixel data over time \\ \hline
\verb#-n arg# & Load up control network for point and camera relationship status \\ \hline
\verb#--additional-pnt-files# & Plot additional point files simultaneously with the above data \\ \hline
\verb#--fullscreen# & Render with the entire screen \\ \hline
\verb#--stereo# & Render in anagylph mode \\ \hline
\verb#--show-moon# & Draw a wireframe moon \\ \hline
\verb#--show-mars# & Draw a wireframe mars \\ \hline
\verb#--show-earth# & Draw a wireframe earth \\ \hline
\end{longtable}


%% \section{reconstruct}
%% \label{reconstruct}

%% Tool under development

%% \section{results}
%% \label{results}

%% Tool under development

%% \section{rmax\_adjust}
%% \label{ramx_adjust}

%% Bundle Adjustment tool specifically for the Yamaha RMAX unmanned
%% aerial vehicle.


