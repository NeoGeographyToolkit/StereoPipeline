\chapter{The VW Log Configuration File}
\label{ch:vwlogconfig}

Remember from before that one of the primary keystones of the Stereo
Pipeline's development is the Vision Workbench. Vision Workbench does
all of the rasterization of images and provides a good many of the
algorithms applied in this software. Since this is the case, Vision
Workbench provides an ability for tuning it's performance and can be
very helpful for low level tuning of the Stereo Pipeline
performance. This chapter is largely a reprint of what is available in
VW's documentation but with additional comments that are helpful for
here.

\section {The {\tt.vwrc} File}

The log configuration file that was alluded to before is the
\verb#.vwrc# file which you can create in your home directory. An
example is provided below. Take a quick look at it as a lot is self
evident.

\begin{minipage}{0.94\linewidth}
\small\listinginput{1}{LogConf.example}
\end{minipage}

\section {Tuning for RAM Considerations}

If the your computer suddenly seems to slow down excessively or you
found yourself staring down an error stating something along the lines
of \verb#St9bad_alloc#, you've ran out of RAM. Don't worry though, the
world is not lost. You can still run the Stereo Pipeline, but just at reduced capacity.

First, it's best to understand the use of RAM inside Vision Workbench
and the Stereo Pipeline. RAM is largely consumed by 2 features, file
cache and rendering.

File Cache is just a memory copy of a file being
read by the program to allow for fast repeated access. The amount of
RAM used for file cache is definable in Vision Workbench log
configuration file under the property \verb#system_cache_size#. The
property is defined in units of bytes. A rule of thumb is to set the
file cache to a quarter of the available system RAM. Also note that if
in a 32 bit OS, the OS can only provide a maximum of 4 GB allocated
per process. In that edge case it's best to use 1 GB.

Memory used by rendering is impossible to restrict to a specific
number as it is a function of many things like processing tile size
and correlation search window size. Yet, the amount of RAM required by
rendering is multiplied by the amount of threads used. It's best to
start out with setting the amount of processing threads equal to
number of logical cores on your system. If you run into a case where
you run out of RAM, then reduce the number of threads until you system
is able to safely run.

\section {Reducing Output on Terminal}

Imagine you are processing with Stereo Pipeline on a cluster and you
are recording the output of stereo for reference. Looking at the
output of the progress bars in the log files can be very annoying. All
progress bars can be shut off by placing under
\verb#[logfile console]# the logging rule:

\begin{verbatim}
0 = *.progress
\end{verbatim}

This will shutoff all progress bars created by the Stereo Pipeline or
by any Vision Workbench tool.
