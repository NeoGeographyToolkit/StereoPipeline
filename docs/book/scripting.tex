\chapter{Modifying SURF to output VW match files}
\label{appendix_surf}

SURF \emph{v1.0.9} is a fast a relatively robust interest point
algorithm. It is not open source, but it is freely available for
academic uses at \url{http://www.vision.ee.ethz.ch/~surf/}. This
software currently only available for Windows and Linux 32 bit.

SURF creates it own results files. What is available online was
probably only meant for demonstrations. What we've done is created a
patch that allows the SURF match utility, \texttt{match.ln}, to create
Vision Workbench match files. The patch is available in
\verb=STEREOPIPELINE/examples/surf_match.patch=.

\section{How to apply and compile}

First move to the directory containing your copy of the SURF v1.0.9
code. Then copy \texttt{surf\_match.patch} to the active directory. At
this point you are ready to start running the following commands.

\begin{verbatim}
        patch < surf_match.patch
        make match.ln
\end{verbatim}

\definecolor{lgray}{gray}{0.95}
\begin{center}
\fcolorbox{black}{lgray}{ \begin{minipage}{5.5in}

    \emph{Note:} \\ If you are unfortunate enough to run into an error
    such as \textit{g++-4.0.2: Command not found}, don't worry. Edit
    \texttt{Makefile} at line 10 and 11 to refer to \texttt{g++}
    instead of \texttt{g++-4.0.2}. \\ \\ Also since you've incurred
    that error, you'll probably need to add an include to
    \texttt{<stdlib.h>} in \texttt{imload.cpp} in the same
    directory. This all stems from differences in using a newer
    version of g++.
\end{minipage}}
\end{center}

\section{Example of using SURF}

For this example it is assumed you have a directory containing two
images named \verb=m1000254.png= and \verb=r0901059.png= like in the
example found in Section \ref{sec:ba_example}.

SURF code only works with images in the grayscale format PGM. A free
Linux utility to convert the images is \texttt{mogrify}. That utility
is part of the package \texttt{imagemagick} and is likely to be
available in most package managers.

Below is the commands to take an input of PNG files, process them with
SURF, and then finally create a match file which can be used by the
likes of \texttt{isis\_adjust}.

\begin{verbatim}
        mogrify -format pgm m1000254.png r0901059.png
        surf.ln -i m1000254.pgm -o m1000254.surf
        surf.ln -i r0901059.pgm -o r0901059.surf
        match.ln -k1 m1000254.surf -k2 r0901059.surf
                 -im1 m1000254.pgm -im2 r0901059.pgm
                 -o out.pgm -m m1000254__r0901059.match
        rm m1000254.pgm r0901059.pgm *.surf
\end{verbatim}

\begin{center}
\fcolorbox{black}{lgray}{ \begin{minipage}{5.5in}

    It is important to note that though SURF is very good at
    performing matches it does not perform a step of RANSAC with its
    output. There may be a couple of outliers.
\end{minipage}}
\end{center}


%% \chapter{Python Batch Processing}

%% Below is a Python script to process a large number of Apollo Metric
%% Camera pairs. This is meant only to serve as an example and it can be
%% modifed to run other commands.

%% \begin{center}\begin{minipage}{5.5in}
%% \begin{Verbatim}[frame=lines,fontsize=\small]
%%     #!/usr/bin/python

%%     import glob;
%%     import os;
%%     import subprocess;

%%     num_cores = 4;
%%     joblist = [];
%%     output_dir = "stereo";

%%     def add_job( job ):
%%         if (len(joblist) >= num_cores):
%%             joblist[0].wait();
%%             joblist.pop(0);
%%         print job;
%%         joblist.append(subprocess.Popen(job,shell=True))
%%         return;

%%     def run_cmd( left_file, right_file ):
%%         h_l = left_file.split(".");
%%         h_r = right_file.split(".");

%%         cmd = "stereo "+left_file+" "+right_file+" "+
%%               h_l[0]+".isis_adjust "+h_r[0]+".isis_adjust "+
%%               output_dir+"/"+h_l[0]+"__"+h_r[0]+" --threads 1"
%%         add_job(cmd);


%%     files = glob.glob("*.cub");
%%     files = sorted(files);
%%     for left in range(0, len(files)):

%%         #determing how forward to look
%%         range_right = left + 2;
%%         if ( range_right > len(files) ):
%%             range_right = len(files);

%%         for right in range(left+1, range_right):
%%             run_cmd( files[left], files[right] );

%%     for j in joblist:
%%         j.wait();

%%     print "Done processing stereo pairs"
%% \end{Verbatim}
%% \end{minipage}\end{center}

