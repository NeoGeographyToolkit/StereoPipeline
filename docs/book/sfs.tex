\chapter{Shape-from-Shading}
\label{ch:sfs}

ASP provides a tool, named \texttt{sfs}, that can improve the level of
detail of DEMs created by ASP or any other source using
\textit{shape-from-shading} (SfS). The tool takes as input one or more
camera images, a DEM at roughly the same resolution as the images, and
returns a refined DEM.

\texttt{sfs} works only with ISIS cub images. It has been tested
thoroughly with Lunar LRO NAC datasets, and some experiments were done
with Mars HiRISE images and with pictures from Charon, Pluto's moon. As
seen later in the text, it returns reasonable results on the Moon as far
as $85^\circ$ South.

Currently, \texttt{sfs} is computationally expensive, and is practical
only for DEMs whose width and height are several thousand pixels. It can be 
sensitive to errors in the position and orientation of the cameras, the
accuracy of the initial DEM, and to the value of the two weights it uses.
Yet, with some effort, it can work quite well. 

A tool named \texttt{parallel\_sfs} is provided (section \ref{psfs}) 
that parallelizes \texttt{sfs} using multiple processes (optionally on
multiple machines) by splitting the input DEM into tiles with padding,
running \texttt{sfs} on each tile, and then blending the results. 

The \texttt{sfs} program can model position-dependent albedo, different
exposure values for each camera, shadows in the input images, and regions
in the DEM occluded from the Sun. It can refine the positions and orientations
of the cameras.

The tool works by minimizing the cost function
\begin{equation}\label{cost}
%\begin{multline}\label{cost}
\int\!\! \int \! \sum_k \left[ I_k(\phi)(x, y) - T_k A(x, y)
 R_k(\phi)(x, y) \right]^2\,  
% R_k(\phi)(x, y) \right]^2\,  \\
+ \mu \left\|\nabla^2 \phi(x, y) \right\|^2  
+ \lambda  \left[ \phi(x, y) - \phi_0(x, y) \right]^2
\, dx\, dy.
\end{equation}
%\end{multline}

Here, $I_k(\phi)(x, y)$ is the $k$-th camera image interpolated at
pixels obtained by projecting into the camera 3D points from the terrain
$\phi(x, y)$, $T_k$ is the $k$-th image exposure, $A(x, y)$ is the
per-pixel albedo, $R_k(\phi)(x, y)$ is the reflectance computed from the
terrain for $k$-th image, $\left\|\nabla^2 \phi(x, y) \right\|^2 $ is the sum
of squares of all second-order partial derivatives of $\phi$, $\mu > 0$
is a smoothing term, and $\lambda > 0$ determines how close we should
stay to the input terrain $\phi_0$ (smaller $\mu$ will show more detail
but may introduce some artifacts, and smaller $\lambda$ may allow for
more flexibility in optimization but the terrain may move too far from the input). 

We use either the regular Lambertian reflectance model,
or the Lunar-Lambertian model \cite{mcewen1991photometric}, more specifically as given in
\cite{lohse2006derivation} (equations (3) and (4)).
Also supported is the Hapke model, 
\cite{johnson2006spectrophotometric}, \cite{fernando2013surface},
\cite{hapke2008bidirectional}, \cite{hapke1993opposition}.
Custom values for the coefficients of these models can be passed to the program.

\section{How to get good test imagery}

We obtain the images from
\url{http://wms.lroc.asu.edu/lroc/search} (we search for EDR images of
type NACL and NACR). 

A faster (but not as complete) interface is provided by
\url{http://ode.rsl.wustl.edu/moon/indexproductsearch.aspx}. The related
site
\url{http://ode.rsl.wustl.edu/moon/indextools.aspx?displaypage=lolardr}
can provide LOLA datasets which can be used as (sparse) ground truth.

We advise the following strategy for picking images. First choose a small
longitude-latitude window in which to perform a search for imagery. Pick
two images that are very close in time and with a big amount of overlap
(ideally they would have consecutive orbit numbers).
Those can be passed to ASP's \texttt{stereo} tool to create an initial DEM. Then, search for 
other images close to the center of the maximum overlap of the first
two images. Pick one or more of those, ideally with different illumination
conditions than the first two. Those (together with one of the first two images) can be used for SfS.

To locate the area of spatial overlap, the images can be map-projected (either
with \texttt{cam2map} with a coarse resolution) or with
\texttt{mapproject}, using for example the LOLA DEM as the terrain to project onto, 
or the DEM obtained from running \texttt{stereo} on those images. Then the images can be overlayed
in \texttt{stereo\_gui}. A good sanity check is to examine the shadows in
various images. If they point in different directions in the images and perhaps
also have different lengths, that means that illumination conditions are
different enough, which will help constrain the \texttt{sfs} problem better.

\section{Running sfs at 1 meter/pixel using a single image}

In both this and the next sections we will work with LRO NAC images taken
close to the Lunar South Pole, at a latitude of $85^\circ$ South (the tool was
tested on equatorial regions as well). We will use four images,
M139939938LE, M139946735RE, M173004270LE, and M122270273LE.

We first retrieve the data sets.
\begin{verbatim}
  wget http://lroc.sese.asu.edu/data/LRO-L-LROC-2-EDR-V1.0/\
LROLRC_0005/DATA/SCI/2010267/NAC/M139939938LE.IMG
  wget http://lroc.sese.asu.edu/data/LRO-L-LROC-2-EDR-V1.0/\
LROLRC_0005/DATA/SCI/2010267/NAC/M139946735RE.IMG
  wget http://lroc.sese.asu.edu/data/LRO-L-LROC-2-EDR-V1.0/\
LROLRC_0009/DATA/SCI/2011284/NAC/M173004270LE.IMG
  wget http://lroc.sese.asu.edu/data/LRO-L-LROC-2-EDR-V1.0/\
LROLRC_0002/DATA/MAP/2010062/NAC/M122270273LE.IMG
\end{verbatim}

Then we convert them to ISIS cubes, initialize the SPICE kernels, and
perform radiometric calibration and echo correction. Here are the steps, 
illustrated on the first image:
\begin{verbatim}  
  lronac2isis from = M139939938LE.IMG     to = M139939938LE.cub
  spiceinit from   = M139939938LE.cub
  lronaccal from   = M139939938LE.cub     to = M139939938LE.cal.cub
  lronacecho from  = M139939938LE.cal.cub to = M139939938LE.cal.echo.cub
\end{verbatim}
We rename, for simplicity, the obtained four processed datasets to
A.cub, B.cub, C.cub, and D.cub.

The first step is to run stereo to create an initial guess DEM. We
picked for this the first two of these images. These form a stereo pair,
that is, they have a reasonable baseline and sufficiently close times of
acquisition (hence very similar illuminations). These conditions are
necessary to obtain a good stereo result.
\begin{verbatim}
parallel_stereo --job-size-w 1024 --job-size-h 1024 A.cub B.cub    \
                --left-image-crop-win 0 7998 2728 2696             \
                --right-image-crop-win 0 9377 2733 2505            \
                --threads 16 --corr-seed-mode 1  --subpixel-mode 3 \
                run_full1/run
\end{verbatim}
Next we create a DEM at 1 meter/pixel, which is about the resolution
of the input images. We use the stereographic projection since this
dataset is very close to the South Pole. Then we crop it to the region
we'd like to do SfS on.
\begin{verbatim}
  point2dem -r moon --stereographic --proj-lon 0 \
    --proj-lat -90 run_full1/run-PC.tif
  gdal_translate -projwin -15471.9 150986 -14986.7 150549  \
    run_full1/run-DEM.tif run_full1/run-crop-DEM.tif
\end{verbatim}
This creates a DEM of size $456 \times 410$ pixels.

Then we run \texttt{sfs}:
\begin{verbatim}
  sfs -i run_full1/run-crop-DEM.tif A.cub -o sfs_ref1/run           \
     --reflectance-type 1                                           \
    --smoothness-weight 0.08 --initial-dem-constraint-weight 0.0001 \
    --max-iterations 10 --use-approx-camera-models                  \
    --use-rpc-approximation --crop-input-images  
\end{verbatim}

The smoothness weight is a parameter that needs tuning. If it is too
small, SfS will return noisy results, if it is too large, too much
detail will be blurred. Here we used the Lunar Lambertian model. The meaning of the other
\texttt{sfs} options can be looked up in section \ref{sfs}.

We show the results of running this program in figure
\ref{fig:sfs1}. The left-most figure is the hill-shaded original DEM,
which was obtained by running:
\begin{verbatim}
  hillshade --azimuth 300 --elevation 20 run_full1/run-crop-DEM.tif \
    -o run_full1/run-crop-hill.tif 
\end{verbatim}
The second image is the hill-shaded DEM obtained after running
\texttt{sfs} for 10 iterations.

The third image is, for comparison, the map-projection of A.cub onto the
original DEM, obtained via the command:
\begin{verbatim}
  mapproject --tr 1 run_full1/run-crop-DEM.tif A.cub A_map.tif --tile-size 128
\end{verbatim}
The forth image is the colored absolute difference between the original
DEM and the SfS output, obtained by running:
\begin{verbatim}
  geodiff --absolute sfs_ref1/run-DEM-final.tif run_full1/run-crop-DEM.tif
  colormap --min 0 --max 2 --colormap-style binary-red-blue \
    run-DEM-final__run-crop-DEM-diff.tif
\end{verbatim}
\begin{figure}[h!]
\begin{center}
\includegraphics[width=7in]{images/sfs1.jpg}
\caption[sfs]{An illustration of \texttt{sfs}. The images are, from
  left to right, the original hill-shaded DEM, the hill-shaded DEM obtained
from \texttt{sfs}, the image A.cub map-projected onto the original DEM,
and the absolute difference of the original and final DEM, where the brightest
shade of red corresponds to a 2 meter height difference.}
\label{fig:sfs1}
\end{center}
\end{figure}

It can be seen that the optimized DEM provides a wealth of detail and
looks quite similar to the input image. It also did not diverge
significantly from the input DEM. We will see in the next section that
SfS is in fact able to make the refined DEM more accurate than the
initial guess (as compared to some known ground truth), though that is
not guaranteed, and most likely did not happen here where just one image
was used.

\section{SfS with multiple images in the presence of shadows}

In this section we will run \texttt{sfs} with multiple images. We would
like to be able to see if SfS improves the accuracy of the DEM rather
than just adding detail to it. We evaluate this using the following
(admittedly imperfect) approach. We resample the original images by a
factor of 10, run stereo with them, followed by SfS using the stereo
result as an initial guess and with the resampled images. As ground
truth, we create a DEM from the original images at 1 meter/pixel, which
we bring closer to the initial guess for SfS using
\texttt{pc\_align}. We would like to know if running SfS brings us even
closer to this ``ground truth'' DEM.

The most significant challenge in running SfS with multiple images is
that shape-from-shading is highly sensitive to errors in camera position
and orientation. The \texttt{sfs} tool can improve these by floating
them during optimization and by using a coarse-to-fine scheme, where
the problem is first solved using subsampled images and terrain then
it is successively refined.

If possible, it may still be desirable to bundle-adjust the cameras first
(section \ref{bundleadjust}). It is important to note that bundle adjustment may
fail if the images have sufficiently different illumination, as it will
not be able to find matches among images. A solution to this is discussed
in section \ref{sfs-lola}.

To make bundle adjustment and stereo faster, we first crop the images,
such as shown below (the crop parameters can be determined via
\texttt{stereo\_gui}).
\begin{verbatim}
  crop from = A.cub to = A_crop.cub sample = 1 line = 6644 nsamples = 2192 nlines = 4982
  crop from = B.cub to = B_crop.cub sample = 1 line = 7013 nsamples = 2531 nlines = 7337
  crop from = C.cub to = C_crop.cub sample = 1 line = 1 nsamples = 2531 nlines = 8305
  crop from = D.cub to = D_crop.cub sample = 1 line = 1 nsamples = 2531 nlines = 2740
\end{verbatim}
Then we bundle-adjust and run stereo
\begin{verbatim}
  bundle_adjust A_crop.cub B_crop.cub C_crop.cub D_crop.cub    \
    --min-matches 1 -o run_ba/run
  stereo A_crop.cub B_crop.cub run_full2/run --subpixel-mode 3 \
    --bundle-adjust-prefix run_ba/run
\end{verbatim}

This will result in a point cloud, \verb#run_full2/run-PC.tif#, which will
lead us to the ``ground truth'' DEM. As mentioned
before, we'll in fact run SfS with images subsampled by a factor of
10. Subsampling is done by running the ISIS \texttt{reduce} command
\begin{verbatim}
  for f in A B C D; do 
    reduce from = ${f}_crop.cub to = ${f}_crop_sub10.cub sscale = 10 lscale = 10
  done
\end{verbatim}

We run bundle adjustment and stereo with the subsampled images using
commands analogous to the above:
\begin{verbatim}
  bundle_adjust A_crop_sub10.cub B_crop_sub10.cub C_crop_sub10.cub D_crop_sub10.cub \
    --min-matches 1 -o run_ba_sub10/run --ip-per-tile 100000
 stereo A_crop_sub10.cub B_crop_sub10.cub run_sub10/run --subpixel-mode 3           \
   --bundle-adjust-prefix run_ba_sub10/run
\end{verbatim}
We'll obtain a point cloud named \verb#run_sub10/run-PC.tif#.

We'll bring the ``ground truth'' point cloud closer to the initial guess
for SfS using \texttt{pc\_align}:
\begin{verbatim}
  pc_align --max-displacement 200 run_full2/run-PC.tif run_sub10/run-PC.tif \
    -o run_full2/run --save-inv-transformed-reference-points
\end{verbatim}
This step is extremely important. Since we ran two bundle adjustment
steps, and both were without ground control points, the resulting clouds
may differ by a large translation, which we correct here. Hence we would like to 
make the ``ground truth'' terrain aligned with the datasets on which we 
will perform SfS. 

Next we create the ``ground truth'' DEM from the aligned high-resolution
point cloud, and crop it to a desired region:
\begin{verbatim}
  point2dem -r moon --tr 10 --stereographic --proj-lon 0 --proj-lat -90 \
    run_full2/run-trans_reference.tif
  gdal_translate -projwin -15540.7 151403 -14554.5 150473               \
    run_full2/run-trans_reference-DEM.tif run_full2/run-crop-DEM.tif
\end{verbatim}
We repeat the same steps for the initial guess for SfS:
\begin{verbatim}
  point2dem -r moon --tr 10 --stereographic --proj-lon 0 --proj-lat -90 \
    run_sub10/run-PC.tif
  gdal_translate -projwin -15540.7 151403 -14554.5 150473               \
    run_sub10/run-DEM.tif run_sub10/run-crop-DEM.tif
\end{verbatim}
After this, we run \texttt{sfs} itself. Since our dataset has many shadows, we found
that specifying the shadow thresholds for the tool improves the
results. The thresholds can be determined using \texttt{stereo\_gui}.
This can be done by turning on shadow-threshold mode from the GUI menu, 
and then clicking on a few points in the shadows.  Then the thresholded images
can be visualized/updated from the menu as well, and this process can be iterated.
\begin{verbatim}
  sfs -i run_sub10/run-crop-DEM.tif A_crop_sub10.cub C_crop_sub10.cub \
    D_crop_sub10.cub -o sfs_sub10_ref1/run --threads 4                \
    --smoothness-weight 0.12 --initial-dem-constraint-weight 0.0001   \
    --reflectance-type 1 --float-exposure                             \
    --float-cameras --use-approx-camera-models                        \
    --max-iterations 10 --use-approx-camera-models                    \
    --use-rpc-approximation --crop-input-images                       \
    --bundle-adjust-prefix run_ba_sub10/run                           \
    --shadow-thresholds "0.00162484 0.0012166 0.000781663"
\end{verbatim}
We compare the initial guess to \texttt{sfs} to the ``ground truth'' DEM
obtained earlier and the same for the final refined DEM using
\texttt{geodiff} as in the previous section. Before SfS:

\begin{verbatim}
  geodiff --absolute run_full2/run-crop-DEM.tif run_sub10/run-crop-DEM.tif
  gdalinfo -stats run-crop-DEM__run-crop-DEM-diff.tif | grep Mean=  
\end{verbatim}
and after SfS:
\begin{verbatim}
  geodiff --absolute run_full2/run-crop-DEM.tif sfs_sub10_ref1/run-DEM-final.tif
  gdalinfo -stats run-crop-DEM__run-DEM-final-diff.tif | grep Mean=
\end{verbatim}

The mean error goes from 2.64~m to 1.29~m, while the standard deviation
decreases from 2.50~m to 1.29~m. Visually the refined DEM looks more detailed
as well as seen in figure \ref{fig:sfs2}. The same experiment can be
repeated with the Lambertian reflectance model (reflectance-type 0), and
then it is seen that it performs a little worse. 

We also show in this figure the first of the images used for SfS,
\verb#A_crop_sub10.cub#, map-projected upon the optimized DEM. Note that
we use the previously computed bundle-adjusted cameras when
map-projecting, otherwise the image will show as shifted from its true
location:
\begin{verbatim}
  mapproject sfs_sub10_ref1/run-DEM-final.tif A_crop_sub10.cub A_crop_sub10_map.tif \
    --bundle-adjust-prefix run_ba_sub10/run
\end{verbatim}
\begin{figure}[h!]
\begin{center}
\includegraphics[width=7in]{images/sfs2.jpg}
\caption[sfs]{An illustration of \texttt{sfs}. The images are, from
  left to right, the hill-shaded initial guess DEM for SfS, the hill-shaded DEM obtained
from \texttt{sfs}, the ``ground truth'' DEM, and the first of the
images used in SfS map-projected onto the optimized DEM.}
\label{fig:sfs2}
\end{center}
\end{figure}

\section{Dealing with large camera errors and LOLA comparison}
\label{sfs-lola}

SfS is very sensitive to errors in camera positions and
orientations. These can be optimized as part of the problem, but if they
are too far off, the solution will not be correct. In the previous
section we used bundle adjustment to correct these errors, and then we
passed the adjusted cameras to \texttt{sfs}. However, bundle adjustment may
often fail, simply because the illumination conditions can be very
different among the images, and interest point matching may not succeed.

The option \texttt{-\/--coarse-levels \it{int}} can be passed to
\texttt{sfs}, to solve for the terrain using a multi-resolution
approach, first starting at a coarse level, where camera errors have
less of an impact, and then jointly optimizing the cameras and
the terrain at ever increasing levels of resolution. Yet, this may still
fail if the terrain does not have large and pronounced features on the
scale bigger than the errors in the cameras.

The approach that we found to work all the time is to manually select
interest points in the images, as the human eye is much more skilled at
identifying a given landmark in multiple images, even when the lightning
changes drastically. Picking about 4 landmarks in each image is
sufficient. Ideally they should be positioned far from each other, to
improve the accuracy.

Below is one example of how we manually select interest points, run SfS,
and then how we compare to LOLA, which is an independently acquired
sparse dataset of 3D points on the Moon. According to
\cite{smith2011results}, the LOLA accuracy is on the order of 1~m. To
ensure a meaningful comparison of stereo and SfS with LOLA, we resample
the LRO NAC images by a factor of 4, making them nominally 4
m/pixel. This is not strictly necessary, the same exercise can be
repeated with the original images, but it is easier to see the
improvement due to SfS when comparing to LOLA when the images are
coarser than the LOLA error itself.

We work with the same images as before. To resample them, we do:
\begin{verbatim}
  for f in A B C D; do 
    reduce from = ${f}_crop.cub to = ${f}_crop_sub4.cub sscale=4 lscale=4
  done
\end{verbatim}

We run \texttt{stereo} and \texttt{point2dem} to get a first cut DEM. We don't do bundle
adjustment at this stage yet. 
\begin{verbatim}
  stereo A_crop_sub4.cub B_crop_sub4.cub run_stereo_noba_sub4/run --subpixel-mode 3
  point2dem --stereographic --proj-lon -5.7113451 --proj-lat -85.000351 \
    run_stereo_noba_sub4/run-PC.tif --tr 4 
\end{verbatim}

We would like now to manually pick interest points for the purpose of
doing bundle adjustment.  We found it it much easier to locate the landmarks if we 
first map-project the images, which brings them all into the same
perspective. We then pick interest points in \texttt{stereo\_gui}, and then project
them back into the original cameras and do bundle adjustment. Here are
the steps:
\begin{verbatim}
  for f in A B C D; do 
    mapproject --tr 4 run_stereo_noba_sub4/run-DEM.tif ${f}_crop_sub4.cub \
      ${f}_crop_sub4_v1.tif --tile-size 128
  done
  stereo_gui A_crop_sub4_v1.tif B_crop_sub4_v1.tif C_crop_sub4_v1.tif \
    D_crop_sub4_v1.tif run_ba_sub4/run
\end{verbatim}
Interest points are selected by zooming and right-clicking with the
mouse, one point at a time, from left to right, and then saving them. An
illustration is shown in Figure \ref{fig:sfs3}.

\begin{figure}[t!]
\begin{center}
\includegraphics[width=7in]{images/sfs3.jpg}
\caption[sfs]{An illustration of how interest points are picked manually for the purpose of bundle adjustment and then SfS.}
\label{fig:sfs3}
\end{center}
\end{figure}

Then bundle adjustment happens:
\begin{verbatim}
  P='A_crop_sub4_v1.tif B_crop_sub4_v1.tif' # to avoid long lines below
  Q='C_crop_sub4_v1.tif D_crop_sub4_v1.tif run_stereo_noba_sub4/run-DEM.tif'
  bundle_adjust A_crop_sub4.cub B_crop_sub4.cub C_crop_sub4.cub D_crop_sub4.cub  \
    -o run_ba_sub4/run --mapprojected-data  "$P $Q"                              \
    --min-matches 1
\end{verbatim}

A good sanity check to ensure that at this stage cameras are aligned properly is to map-project
using the newly obtained camera adjustments and then overlay the obtained images in the GUI.
The features in all images should be perfectly on top of each other.
\begin{verbatim}
  for f in A B C D; do 
    mapproject --tr 4 run_stereo_noba_sub4/run-DEM.tif ${f}_crop_sub4.cub  \
     ${f}_crop_sub4_v2.tif --tile-size 128 --bundle-adjust-prefix run_ba_sub4/run
  done
\end{verbatim}
This will also show where the images overlap, and hence on what portion of the DEM we
can run SfS.

Then we run stereo, followed by SfS. 
\begin{verbatim}
  stereo A_crop_sub4.cub B_crop_sub4.cub run_stereo_yesba_sub4/run             \
    --subpixel-mode 3 --bundle-adjust-prefix run_ba_sub4/run
  point2dem --stereographic --proj-lon -5.7113451 --proj-lat -85.000351        \
    run_stereo_yesba_sub4/run-PC.tif --tr 4
  gdal_translate -srcwin 138 347 273 506 run_stereo_yesba_sub4/run-DEM.tif     \
    run_stereo_yesba_sub4/run-crop1-DEM.tif 
  sfs -i run_stereo_yesba_sub4/run-crop1-DEM.tif A_crop_sub4.cub               \
    C_crop_sub4.cub D_crop_sub4.cub -o sfs_sub4_ref1_th_reg0.12_wt0.001/run    \
    --shadow-thresholds '0.00149055 0.00138248 0.000747531'                    \
    --threads 4 --smoothness-weight 0.12 --initial-dem-constraint-weight 0.001 \
    --reflectance-type 1 --float-exposure --float-cameras --max-iterations 20  \
    --use-approx-camera-models --use-rpc-approximation --crop-input-images     \
    --bundle-adjust-prefix run_ba_sub4/run
\end{verbatim}

We fetch the portion of the LOLA dataset around the current DEM from the
site described earlier, and save it as
\verb#RDR_354E355E_85p5S84SPointPerRow_csv_table.csv#. It is necessary
to align our stereo DEM with this dataset to be able to compare
them. We choose to bring the LOLA dataset into the coordinate system
of the DEM, using:
\begin{verbatim}
  pc_align --max-displacement 280 run_stereo_yesba_sub4/run-DEM.tif             \
    RDR_354E355E_85p5S84SPointPerRow_csv_table.csv -o run_stereo_yesba_sub4/run \
    --save-transformed-source-points
\end{verbatim}

Then we compare to the aligned LOLA dataset the input to SfS and its output:
\begin{verbatim}
  geodiff --absolute -o beg --csv-format '1:lon 2:lat 3:radius_km' \
    run_stereo_yesba_sub4/run-crop1-DEM.tif run_stereo_yesba_sub4/run-trans_source.csv
  geodiff --absolute -o end --csv-format '1:lon 2:lat 3:radius_km' \
    sfs_sub4_ref1_th_reg0.12_wt0.001/run-DEM-final.tif             \
    run_stereo_yesba_sub4/run-trans_source.csv
\end{verbatim}

We see that the mean error between the DEM and LOLA goes down, after SfS,
from  1.14~m to 0.90~m, while the standard deviation decreases from
 1.18~m to 1.06~m.

\section{Running SfS with an external initial guess DEM}

Sometimes it is convenient to run SfS with a DEM not created using ASP's
\texttt{stereo}. For example, for the Moon, the LOLA gridded DEM is
available. It is somewhat noisy and the nominal resolution is on the
order of 10 m/pixel, but it is available even for permanently shadowed
regions.

The main challenge in such a situation is that the images, such as
coming from LRO NAC, may not be aligned well to this external DEM and
among themselves, and then SfS will fail. To get an initial alignment,
what worked for us to coarsen both this DEM and the images to about 40
meters/pixel (using the ISIS \texttt{reduce} command and
\texttt{dem\_mosaic} with the options to blur and change the grid size),
and then run SfS over a reasonably large area (say about $500 \times
500$ pixels) with pronounced terrain and where images have a lot of
overlap with the option \texttt{-\/-float-all-cameras} (and of course
\texttt{-\/-float-exposure}, and an appropriate \texttt{-\/-shadow-threshold}). 
Then, the SfS program will find
adjustments to the cameras, writing them in the output directory. The
\texttt{mapproject} tool can be used to map-project the coarse images
onto the input DEM using \texttt{-\/-bundle-adjust-prefix} pointing to
the \texttt{sfs} output prefix. This should be used to verify that images
are now aligned correctly, for example by overlaying them in
\texttt{stereo\_gui}. If so, these adjustments can be used as input for
SfS with images at finer levels of resolution (after appropriately renaming the
adjustment files and using the \texttt{-\/-bundle-adjust-prefix} option of \texttt{sfs}).

Here, a higher value can be used for \texttt{-\/-initial-dem-constraint-weight}
to ensure we cameras have more motivation to align to the terrain.

When it comes to running SfS with many large high-resolution
images, one runs very fast into memory constraints. It is then necessary
to parallelize the problem using \texttt{parallel\_sfs}, which will run
\texttt{sfs} on multiple tiles. Yet, the camera adjustments need to be
determined before running this tool, and then kept fixed when this tool
is run, as otherwise each single tile will optimize its cameras
independently, and as result there will be discontinuities at tile
boundaries. 

If the camera adjustments are determined by first running \texttt{sfs}
on a clip representative of the entire terrain, far from that clip the
cameras will start to disagree. For that reason, \texttt{sfs} makes it
possible to optimize the cameras on an entire collection of clips,
chosen so they are reasonably spread out over the entire terrain, 
and that each camera image is covered by at least a handful of such clips.
The clips can be passed to \texttt{sfs} as a quoted string via the \texttt{-i} option. 
As before, at the end the \texttt{mapproject}
program can be used to verify that the camera images mapprojected using the
obtained camera adjustments are perfectly on top of each other, and if not,
more clips can be added to the joint optimization problem. When a good
enough set of camera adjustments is obtained, \texttt{parallel\_sfs} can
be run as before. 

\section{Insights for getting the most of SfS}
\label{sfs:insights}

Here are a few suggestions we have found helpful when running \texttt{sfs}:

\begin{itemize}{}

\item First determine the appropriate smoothing weight $\mu$ by running a
small clip, and using just one image. A value between 0.06 and 0.12 seems to work 
all the time with LRO NAC, even when the images are subsampled. The other weight, $\lambda,$ can
be set to something small, like $0.0001.$ This can be increased to $0.001$ if noticing
that the output DEM strays too far. 

\item As stated before, more images with more diverse illumination conditions
result in more accurate terrain. Ideally there should be at least 3 images, with the shadows
being, respectively, on the left, right, and then perhaps missing or small. 

\item Bundle-adjustment for multiple images is crucial, to eliminate
  camera errors which will result in \texttt{sfs} converging to a local
  minimum. This is described in section \ref{sfs-lola}.

\item Floating the albedo (option \texttt{-\/-float-albedo}) can
  introduce instability and divergence, it should be avoided unless
  obvious albedo variation is seen in the images.

\item Floating the DEM at the boundary (option
  \texttt{-\/-float-dem-at-boundary}) is also suggested to be avoided.

\item Overall, the best strategy is to first use SfS for a single image
  and not float any variables except the DEM being optimized, and then
  gradually add images and float more variables and select whichever
  approach seems to give better results.

\item If an input DEM is large, it may not be completely covered by a
single set of imagery with various illumination conditions.  It should
then be broken up into smaller regions (with overlap), the SfS problem
can be solved on each region, and then every output terrain can be
transformed using \texttt{pc\_align} into LOLA's global coordinate
system, where they can be mosaicked together using
\texttt{dem\_mosaic}. Or, \texttt{sfs} can be run not on one clip, but
on an entire collection of clips covering this area to get the
adjustments, and then \texttt{parallel\_sfs} can be run as described in
the previous section. 

  The easier case is when at least the two images in the stereo pair
  cover the entire terrain. Then, portions of this terrain can be used
  as an initial guess for each SfS sub-problem (even as the other images
  used for SfS change), the results can be mosaicked, and the alignment
  to LOLA can happen just once, after mosaicking. This approach is
  preferable, if feasible, as alignment to LOLA is more accurate if the
  terrain to align is larger in extent.

\item The \texttt{mapproject} program can be used to map-project each image
onto the resulting SfS DEM (with the camera adjustments solved using
SfS). These orthoimages can be mosaicked using \texttt{dem\_mosaic}. If the
\texttt{-\/-max} option is used with this tool, it create a mosaic with
the most illuminated pixels from this image. If during SfS the camera
adjustments were solved accurately, this mosaic should have little or no
blur. An alterantive is to use the \texttt{-\/-block-max} option
which will pick the most lit of the images per each block, with the latter
being specified via \texttt{-\/-block-size}. 
\end{itemize}
