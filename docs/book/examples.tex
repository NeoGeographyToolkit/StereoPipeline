\chapter{Example Data Processing}

This chapter is partly a show case of results that are possible with
Ames Stereo Pipeline. Yet it is also a shortened guide that shows the
commands and stereo.default files used to process data. It can be hard
try to figure out what settings to start with and hopefully this will
provide starting point ideas.

\section{Apollo 15 Metric Camera}

\subsection{Ansgarius C}

\subsubsection*{Screenshot}

\begin{figure}[hb]
\centering
  \subfigure[{\tt 3D Rendering}]{\includegraphics[width=3in]{images/examples/metric/metric_example.png}}
  \hfil
  \subfigure[{\tt KML Screenshot}]{\includegraphics[width=3in]{images/examples/metric/metric_ge_example.png}}
\caption{Example output possible with Apollo Metric frames.}
\label{fig:metric_example}
\end{figure}

\subsubsection*{Commands}

\begin{verbatim}
    % Process tif files with not yet released commands %
    reduce from=AS15-M-2380.cub to=sub4-AS15-M-2380.cub sscale=4 lscale=4
    reduce from=AS15-M-2381.cub to=sub4-AS15-M-2381.cub sscale=4 lscale=4
    spiceinit from=sub4-AS15-M-2380.cub
    spiceinit from=sub4-AS15-M-2381.cub
    ipfind --max 10000 sub4*.cub
    ipmatch -i 50 -r homography sub4*.cub
    mkdir result
    stereo sub4-AS15-M-2380.cub sub4-AS15-M-2381.cub result/output
\end{verbatim}

\subsubsection*{Stereo Default}

\begin{verbatim}
    ##      PREPROCESSING      ##

    DO_INTERESTPOINT_ALIGNMENT 1
    DO_EPIPOLAR_ALIGNMENT 0
    INTERESTPOINT_ALIGNMENT_SUBSAMPLING 0
    FORCE_USE_ENTIRE_RANGE 1

    PREPROCESSING_FILTER_MODE 3
    SLOG_KERNEL_WIDTH 1.5

    ###########################    CORRELATION    ###########################

    H_KERNEL 35
    V_KERNEL 35
    SUBPIXEL_H_KERNEL 25
    SUBPIXEL_V_KERNEL 25

    H_CORR_MIN -250
    H_CORR_MAX 250
    V_CORR_MIN -70
    V_CORR_MAX 100

    SUBPIXEL_MODE 3
    DO_H_SUBPIXEL 1
    DO_V_SUBPIXEL 1

    XCORR_THRESHOLD 2.0
    CORRSCORE_REJECTION_THRESHOLD 1.4

    COST_BLUR 25
    COST_MODE 2

    ############################    FILTERING    ############################

    FILL_HOLES 1
    MASK_FLATFIELD 1

    RM_H_HALF_KERN 5
    RM_V_HALF_KERN 5
    RM_MIN_MATCHES 60 # Units = percest
    RM_THRESHOLD 3

    #############################    DOTCLOUD    ############################

    NEAR_UNIVERSE_RADIUS 0.0
    FAR_UNIVERSE_RADIUS 0.0

\end{verbatim}

\subsubsection*{Comments}

Apollo Metric images were all taken at regular intervals, that means
that the same stereo.default can be used for all sequential pairs of
images. Apollo Metric images are some of easiest images to make stereo
data from.

The scans performed by ASU are absurdly detailed to the extent that the
film grain can be made out. This detail of information is not helpful
for the correlator, so it is recommended that subsampling the image by
4 so an actual signal is available at 1 px samplings.

\section{Cassini ISS NAC}

\subsection{Enceladus}

\subsubsection*{Screenshot}

text

\subsubsection*{Commands}

text

\subsubsection*{Stereo Default}

text

\section{Lunar Reconaissance Orbiter LROC-NA}

\subsection{Lincoln Scarp}

\subsubsection*{Screenshot}

text

\subsubsection*{Commands}

text

\subsubsection*{Stereo Default}

text

\section{Mars Global Surveyor MOC-NA}

\subsection{North Terra Meridiani}

\subsubsection*{Screenshot}

text

\subsubsection*{Commands}

text

\subsubsection*{Stereo Default}

text

\section{Mars Reconaissance Orbiter CTX}

\subsection{North Terra Meridiani}

\subsubsection*{Screenshot}

text

\subsubsection*{Commands}

text

\subsubsection*{Stereo Default}

text

\section{Mars Reconaissance Orbiter HiRISE}

\subsection{Columbia Hills}

\subsubsection*{Screenshot}

text

\subsubsection*{Commands}

text

\subsubsection*{Stereo Default}

text

\subsection{Victoria Crater}

\subsubsection*{Screenshot}

text

\subsubsection*{Commands}

text

\subsubsection*{Stereo Default}

text

\section{MESSENGER MDIS}

\subsection{Wide Angle on flyby 2}

\subsubsection*{Screenshot}

\begin{figure}[ht]
  \begin{center}
  \includegraphics[width=5in]{images/examples/mdis/mdis_wide_example.png}
  \end{center}
  \caption{ A rough attempt at MDIS imagery }
  \label{fig:mdis_attempt}
\end{figure}

\subsubsection*{Commands}

\begin{verbatim}
    mdis2isis from=EW0108825359A.IMG to=EW0108825359A.cub
    mdis2isis from=EW0108825379C.IMG to=EW0108825379C.cub
    spiceinit from=EW0108825359A.cub
    spiceinit from=EW0108825359C.cub
    ipfind --max 10000 *.cub
    ipmatch -i 10 -r homography *.cub
    mkdir result
    stereo EW0108825359A.cub EW0108825379C.cub stereo/output
\end{verbatim}

\subsubsection*{Stereo Default}

\subsubsection*{Comments}

This a proof of concept. Like most imagery coming from spacecraft that
are currently not in orbit with their target, it is very hard to find
good stereo pairs. This is taken from a single flyby from the same
camera seconds apart. Unfortunately theres not enough of a perspective
change to make anything other than the spherical surface.

For future mappers, flyby 3 happen recently. It seems flyby 3 and
flyby 2 could have covered some of the same terrain with the narrow
angle camera. Working with the narrow angle should provide a better
stereo pair as our software seems is tuned to larger kernels.


