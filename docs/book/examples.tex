\chapter{Data Processing Examples}
\label{ch:examples}

This chapter showcases a variety of results that are possible when
processing different data sets with the Stereo Pipeline. It is also a
shortened guide that shows the commands and stereo.default files used
to process data. We hope that these are useful templates that will get
you started in processing your own data.

\section{Guidelines for Selecting Stereo Pairs}

When choosing image pairs to process, images that are taken with
similar viewing angles, lighting conditions, and significant surface
coverage overlap are best suited for creating terrain
models. Depending on the characteristics of the mission data set and
the individual images, the degree of acceptable variation will
differ. Significant differences between image characteristics
increases the liklihood of stereo matching error and artifacts, and
these errors will propagate through to the resulting data products.

Although images do not need to be map projected before running the
\texttt{stereo} program, we recommend that you do run {\tt cam2map}
beforehand, especially for image pairs that contain large topographic
variation (and therefore large disparity differences across the scene,
\emph{e.g. Valles Marineris}).  Map projection is especially necessary
when processing HiRISE images. This removes the large disparity
differences between HiRISE images and leaves only the small detail for
the Stereo Pipeline to compute. Remember that ISIS can work backwards
through a map-projection when applying the camera model, so the
geometric integrity of your images will not be sacrified if you map
project first.

Excessively noisy images will not correlate well, so images should be
photometrically calibrated in whatever fashion suits your purposes. If
there are photometric problems with the images, those photometric
defects can be misinterpreted as topography.

Remember, in order for \texttt{stereo} to process stereo pairs in ISIS
CUBE format, the images must have had SPICE data associated by running
ISIS's \texttt{spiceinit} program run on them first.

\section{Mars Reconaissance Orbiter HiRISE}

HiRISE is one of the most challenging cameras to use when making 3D
models because HiRISE exposures can be several gigabytes each. Working
with this data requires patience as it will take time.

One important fact to know about HiRISE is that it is composed of
multiple linear CCDs that are arranged side by side with some vertical
offsets. These offsets mean that the CCDs will view some of the same
terrain but at a slightly different time and a slightly different
angle. Mosiacing the CCDs together to a single image is not a simple
process and involves living with some imperfections.

At this time, we recommend mosaicking CCDs using the scripts available
at your institution (both USGS and University of Arizona have this
capability).  We are also providing a script that will take raw HiRISE
images and combine them into a mosaic. Our script takes all red CCDs
and projects them using the ISIS {\tt noproj} command into the
perspective of RED5 CCD. From there, {\tt hijitreg} is performed to
work out the relative offsets between CCDs. Finally the CCDs are
mosaicked together using the average offset listed from {\tt hijitreg}
using the {\tt handmos} command. Below is an outline of what our
script does.  It is based on the tutorial provided on the USGS ISIS
website.

\begin{verbatim}
    command outline
\end{verbatim}

Finally we recommend map projecting the product and normalizing both
images in the stereo pair using the same dynamic range. Notice that we
map project the second image using the same map settings and crop of
the first image. This means the images will share the same origin and
the {\tt stereo.default} search range can be centered around zero.

\begin{verbatim}
    cam2map f=first.cub t=first.map.cub
    cam2map f=second.cub map=first.map.cub t=second.map.cub matchmap=true
    bandnorm f=first.map.cub t=first.norm.cub
    bandnorm f=second.map.cub t=second.norm.cub
    ls first.norm.cub second.norm.cub > fromlist
    ls first.norm.cub > holdlist
    equalizer fromlist=fromlist holdlist=holdlist
    mkdir result
    stereo first.norm.equ.cub second.norm.equ.cub result/output
\end{verbatim}

In the future, it is our understanding that the HiRISE team will be
producing stitch but non-map--projected imagery to the PDS. If this
happens, most of the above commands will no longer be required.

\subsection{Columbia Hills}

The following description is taken and shortened from the HiRISE
instrument website.
\url{http://hirise.lpl.arizona.edu/PSP_001513_1655}

\begin{quotation}
This HiRISE image shows the landing site of the Mars Exploration Rover
Spirit. The impact crater in the upper left-hand portion of the image
is "Bonneville Crater," which was investigated by Spirit shortly after
landing. In the lower right-hand portion of the image is "Husband
Hill," a large hill that Spirit climbed and where it spent much of its
now nearly three-year mission.

The bright irregularly-shaped feature in the area north-west of
Bonneville Crater is Spirit's parachute, now lying on the Martian
surface. Near the parachute is the cone-shaped "backshell" that helped
protect Spirit's lander during its seven-month journey to Mars. The
backshell appears relatively undamaged by its impact with the martian
surface. Wrinkles and folds in the parachute fabric are clearly
visible.

Immediately south west of Bonneville Crater shows Spirit's lander. The
crater in the upper left-hand portion of the image, just to the
northwest of the lander, is the one that the Mars Exploration Rover
team named "Sleepy Hollow."

The north rim of Bonneville Crater shows the damaged remnant of the
heat shield that protected the vehicle during the high-speed entry
through the Martian atmosphere. The heat shield impacted the surface
after being separated from the vehicle during the final stages of the
descent.

South of Husband Hill shows the current location of Spirit. Toward the
top of the image is "Home Plate," a plateau of layered rocks that
Spirit explored during the early part of its third year on
Mars. Spirit itself is clearly seen just to the southeast of Home
Plate. Also visible are the tracks made by the rover before it arrived
at its current location.

Written by: Steve Squyres
\end{quotation}

\subsubsection*{Screenshot}

\begin{figure}[h!]
\centering
  \subfigure[{\tt 3D Rendering}]{\includegraphics[width=3in]{images/examples/hirise/chills_hirise_example.png}}
  \hfil
  \subfigure[{\tt KML Screenshot}]{\includegraphics[width=3in]{images/examples/hirise/chills_hirise_ge_example.png}}
\caption{Example output using HiRISE images PSP\_001513\_1655 and
  PSP\_001777\_1650 of East Mareotis Tholus.}
\label{fig:hirise_chills_example}
\end{figure}

\subsubsection*{Commands}

\begin{verbatim}
    % Download all of the RED IMG for PSP_001513_1655 & %
    %                                 PSP_001777_1650   %
    ~/HiRISE_stitch.py PSP_001513_1655_RED0_0.IMG
    ~/HiRISE_stitch.py PSP_001777_1650_RED0_0.IMG
    cam2map from=PSP_001513_1655_REDmosaic.norm.cub to=PSP_001513_1655_REDmosaic.map.cub
    cam2map from=PSP_001777_1650_REDmosaic.norm.cub map=PSP_001513_1655_REDmosaic.map.cub ...
            to=PSP_001777_1650_REDmosaic.norm.cub matchmap=true
    bandnorm from=PSP_001513_1655_REDmosaic.map.cub to=PSP_001513_1655_REDmosaic.map.norm.cub
    bandnorm from=PSP_001777_1650_REDmosaic.map.cub to=PSP_001777_1650_REDmosaic.map.norm.cub
    ls *.map.norm.cub > fromlist
    ls *1513*.map.norm.cyb > holdlist
    equalizer fromlist=fromlist holdlist=holdlist
    rm *REDmosaic.map.norm.cub *REDmosaic.map.cub
    mkdir result
    stereo PSP_001513_1655.map.norm.equ.cub PSP_001777_1650.map.norm.equ.cub result/output
\end{verbatim}

\subsubsection*{Stereo Default}

\begin{verbatim}
    ### PREPROCESSING

    DO_INTERESTPOINT_ALIGNMENT 0
    INTERESTPOINT_ALIGNMENT_SUBSAMPLING 0
    DO_EPIPOLAR_ALIGNMENT 0

    FORCE_USE_ENTIRE_RANGE 1
    DO_INDIVIDUAL_NORMALIZATION 0

    PREPROCESSING_FILTER_MODE 2

    SLOG_KERNEL_WIDTH 1.5

    ### CORRELATION

    COST_MODE 0
    COST_BLUR 0

    H_KERNEL 50
    V_KERNEL 50

    H_CORR_MIN 210
    H_CORR_MAX 450
    V_CORR_MIN -320
    V_CORR_MAX 320

    SUBPIXEL_MODE 2

    SUBPIXEL_H_KERNEL 25
    SUBPIXEL_V_KERNEL 25

    ### FILTERING

    FILL_HOLES 1

    RM_H_HALF_KERN 5
    RM_V_HALF_KERN 5
    RM_MIN_MATCHES 60 # Units = percent
    RM_THRESHOLD 3
    RM_CLEANUP_PASSES 1

    ### DOTCLOUD

    NEAR_UNIVERSE_RADIUS 0.0
    FAR_UNIVERSE_RADIUS 0.0
\end{verbatim}

\subsection{East Mareotis Tholus}

The description is taken from the HiRISE instrument website.
\url{http://hirise.lpl.arizona.edu/PSP_001760_2160}

\begin{quotation}
East Mareotis Tholus is a small volcano in Tempe Terra, Mars. This
area is on the northeast edge of the Tharsis bulge that was built up
by many large and small volcanoes.

One of the many questions we hope to address with HiRISE is the
relative roles of the giant shield volcanoes (such as Olympus Mons)
and smaller volcanic features (such as East Mareotis Tholus).

The anaglyph covers 4.4 x 6.9 km (2.7 x 4.9 miles) and the topography
can be viewed using red-blue glasses. The elongated pit at the summit
of the volcano is where the lava issued forth. The large circular hole
just to the SW of the vent is an impact crater. The gouges in the
ground to the SE of the volcano are tectonic fissures (called graben)
that are now filled with sand dunes. The area is covered with large
amounts of wind-blown dust, so it is not surprising that lava flows
and other smaller volcanic features are not visible.

However, the smooth shape of the volcano, and the lack of lava layers
exposed in the impact crater, allow for the possiblity that this
volcano is composed largely of ash, rather than lava flows.

Written by: Laszlo P. Keszthelyi
\end{quotation}

\subsubsection*{Screenshot}

\begin{figure}[h!]
\centering
  \subfigure[{\tt 3D Rendering}]{\includegraphics[width=3in]{images/examples/hirise/emare_example.png}}
  \hfil
  \subfigure[{\tt KML Screenshot}]{\includegraphics[width=3in]{images/examples/hirise/emare_ge_example.png}}
\caption{Example output using HiRISE images PSP\_001364\_2160 and
  PSP\_001760\_2160 of East Mareotis Tholus.}
\label{fig:hirise_emare_example}
\end{figure}

\subsubsection*{Commands}

\begin{verbatim}
    % Download all of the RED IMG for PSP_001364_2160 & %
    %                                 PSP_001760_2160   %
    ~/HiRISE_stitch.py PSP_001364_2160_RED0_0.IMG
    ~/HiRISE_stitch.py PSP_001760_2160_RED0_0.IMG
    cam2map from=PSP_001364_2160_REDmosaic.norm.cub to=PSP_001364_2160_REDmosaic.map.cub
    cam2map from=PSP_001760_2160_REDmosaic.norm.cub map=PSP_001364_2160_REDmosaic.map.cub ...
            to=PSP_001760_2160_REDmosaic.map.cub matchmap=true
    bandnorm from=PSP_001364_2160_REDmosaic.map.cub to=PSP_001364_2160_REDmosaic.map.norm.cub
    bandnorm from=PSP_001760_2160_REDmosaic.map.cub to=PSP_001760_2160_REDmosaic.map.norm.cub
    ls *.map.norm.cub > fromlist
    ls *1760*.map.norm.cub > holdlist
    equalizer fromlist=fromlist holdlist=holdlist
    rm *REDmosaic.map.norm.cub *REDmosaic.map.cub
    mkdir result
    stereo PSP_001364_2160.map.norm.equ.cub PSP_001760_2160.map.norm.equ.cub result/output
\end{verbatim}

\subsubsection*{Stereo Default}

\begin{verbatim}
    ### PREPROCESSING

    DO_INTERESTPOINT_ALIGNMENT 0
    INTERESTPOINT_ALIGNMENT_SUBSAMPLING 0
    DO_EPIPOLAR_ALIGNMENT 0

    FORCE_USE_ENTIRE_RANGE 1
    DO_INDIVIDUAL_NORMALIZATION 0

    PREPROCESSING_FILTER_MODE 2

    SLOG_KERNEL_WIDTH 1.5

    ### CORRELATION

    COST_BLUR 0
    COST_MODE 0

    H_KERNEL 25
    V_KERNEL 25

    H_CORR_MIN -80
    H_CORR_MAX 150
    V_CORR_MIN -80
    V_CORR_MAX 50

    SUBPIXEL_MODE 0

    SUBPIXEL_H_KERNEL 25
    SUBPIXEL_V_KERNEL 25

    ### FILTERING

    FILL_HOLES 1

    RM_H_HALF_KERN 5
    RM_V_HALF_KERN 5
    RM_MIN_MATCHES 60 # Units = percest
    RM_THRESHOLD 3
    RM_CLEANUP_PASSES 1

    ### DOTCLOUD

    NEAR_UNIVERSE_RADIUS 0.0
    FAR_UNIVERSE_RADIUS 0.0
\end{verbatim}

\subsection{North Terra Meridiani Crop}

HiRISE website only has to say that this is `Layered Materials within
a Small Crater'. Hopefully you'll still agree that this is cool.

\subsubsection*{Screenshot}

\begin{figure}[h!]
\centering
  \subfigure[{\tt 3D Rendering}]{\includegraphics[width=3in]{images/examples/hirise/nterra_example.png}}
  \hfil
  \subfigure[{\tt KML Screenshot}]{\includegraphics[width=3in]{images/examples/hirise/nterra_ge_example.png}}
\caption{Example output using cropped HiRISE data of North Terra Meridiani.}
\label{fig:hirise_nterra_example}
\end{figure}

\subsubsection*{Commands}

Notice here that we have applied a crop to select a subset of these
HiRISE images that we are interested in.  Cropping is often an
efficient way to go because it greatly reduces the amount of
computation necessary to get results in a limited area.

\begin{verbatim}
    % Download all of the IMG for PSP_001981_1825 & %
    %                             PSP_002258_1825   %
    ~/HiRISE_stitch.py PSP_001981_1825_RED0_0.IMG
    ~/HiRISE_stitch.py PSP_002258_1825_RED0_0.IMG
    cam2map from=PSP_001981_1825_REDmosaic.norm.cub to=PSP_001981_1825_REDmosaic.map.cub
    cam2map from=PSP_002258_1825_REDmosaic.norm.cub map=PSP_001981_1825_REDmosaic.map.cub ...
            to=PSP_002258_1825_REDmosaic.map.cub matchmap=true
    bandnorm from=PSP_001981_1825_REDmosaic.map.cub to=PSP_001981_1825_REDmosaic.map.norm.cub
    bandnorm from=PSP_002258_1825_REDmosaic.map.cub to=PSP_002258_1825_REDmosaic.map.norm.cub
    ls *.map.norm.cub > fromlist
    ls *1981*.map.norm.cub > holdlist
    equalizer fromlist=fromlist holdlist=holdlist
    crop from=PSP_001981_1825_REDmosaic.map.norm.equ.cub to=PSP_001981_1825.crop.cub ...
         sample=7497 line=41318 nsamp=10000 nline=10000
    crop from=PSP_002258_1825_REDmosaic.map.norm.equ.cub to=PSP_002258_1825.crop.cub ...
         sample=7982 line=41310 nsamp=10000 nline=10000
    rm *REDmosaic*.cub
    mkdir result
    stereo PSP_001981_1825.crop.cub PSP_002258_1825.crop.cub result/output
\end{verbatim}

\subsubsection*{Stereo Default}

\begin{verbatim}
    ### PREPROCESSING

    DO_INTERESTPOINT_ALIGNMENT 0
    INTERESTPOINT_ALIGNMENT_SUBSAMPLING 0
    DO_EPIPOLAR_ALIGNMENT 0

    FORCE_USE_ENTIRE_RANGE 1
    DO_INDIVIDUAL_NORMALIZATION 0

    PREPROCESSING_FILTER_MODE 2

    SLOG_KERNEL_WIDTH 1.5

    ### CORRELATION

    COST_BLUR 21
    COST_MODE 2

    H_KERNEL 45
    V_KERNEL 45

    H_CORR_MIN -270
    H_CORR_MAX -70
    V_CORR_MIN -14
    V_CORR_MAX 26

    SUBPIXEL_MODE 0

    SUBPIXEL_H_KERNEL 25
    SUBPIXEL_V_KERNEL 25

    ### FILTERING

    FILL_HOLES 1

    RM_H_HALF_KERN 5
    RM_V_HALF_KERN 5
    RM_MIN_MATCHES 60 # Units = percent
    RM_THRESHOLD 3
    RM_CLEANUP_PASSES 1

    ### DOTCLOUD

    NEAR_UNIVERSE_RADIUS 0.0
    FAR_UNIVERSE_RADIUS 0.0
\end{verbatim}


\section{Mars Reconaissance Orbiter CTX}

\subsection{North Terra Meridiani}

\subsubsection*{Screenshot}

\begin{figure}[h!]
\centering
  \subfigure[{\tt 3D Rendering}]{\includegraphics[width=3in]{images/examples/ctx/n_terra_meridiani_ctx.png}}
  \hfil
  \subfigure[{\tt KML Screenshot}]{\includegraphics[width=3in]{images/examples/ctx/n_terra_meridiani_ctx_ge.png}}
\caption{Example output possible with the CTX imager aboard MRO.}
\label{fig:ctx_example}
\end{figure}

\subsubsection*{Commands}

\begin{verbatim}
    % Download P02_001981_1823_XI_02N356W.IMG &
    %          P03_002258_1817_XI_01N356W.IMG
    mroctx2isis from=P02_001981_1823_XI_02N356W.IMG to=P02_001981_1823_XI_02N356W.cub
    mroctx2isis from=P03_002258_1817_XI_01N356W.IMG to=P03_002258_1817_XI_01N356W.cub
    spiceinit from=P02_001981_1823_XI_02N356W.cub
    spiceinit from=P03_002258_1817_XI_01N356W.cub
    ctxcal from=P02_001981_1823_XI_02N356W.cub to=P02_001981_1823_XI_02N356W.cal.cub
    ctxcal from=P03_002258_1817_XI_01N356W.cub to=P03_002258_1817_XI_01N356W.cal.cub
    cam2map from=P02_001981_1823_XI_02N356W.cal.cub to=P02_001981_1823_XI_02N356W.map.cub
    cam2map from=P03_002258_1817_XI_01N356W.cal.cub to=P03_002258_1817_XI_01N356W.map.cub
\end{verbatim}

\subsubsection*{Stereo Default}

\begin{verbatim}
    ### PREPROCESSING

    DO_INTERESTPOINT_ALIGNMENT 1
    INTERESTPOINT_ALIGNMENT_SUBSAMPLING 0
    DO_EPIPOLAR_ALIGNMENT 0

    FORCE_USE_ENTIRE_RANGE 0
    DO_INDIVIDUAL_NORMALIZATION 0

    PREPROCESSING_FILTER_MODE 3

    SLOG_KERNEL_WIDTH 1.5

    ### CORRELATION

    COST_BLUR 0
    COST_MODE 2

    H_KERNEL 35
    V_KERNEL 35

    H_CORR_MIN -300
    H_CORR_MAX 300
    V_CORR_MIN -150
    V_CORR_MAX 150

    SUBPIXEL_MODE 3

    SUBPIXEL_H_KERNEL 21
    SUBPIXEL_V_KERNEL 21

    ### FILTERING

    FILL_HOLES 1

    RM_H_HALF_KERN 5
    RM_V_HALF_KERN 5
    RM_MIN_MATCHES 60 # Units = percest
    RM_THRESHOLD 3
    RM_CLEANUP_PASSES 1

    ### DOTCLOUD

    NEAR_UNIVERSE_RADIUS 0.0
    FAR_UNIVERSE_RADIUS 0.0
\end{verbatim}

\section{Mars Global Surveyor MOC-NA}

In the Stereo Pipeline Tutorial in Chapter~\ref{ch:tutorial}, we
showed you how to process a MOC-NA stereo pair that covered the
Galaxius Fluctus channel. In this section we will show you more
examples, some of which exhibit a problem common to stereo pairs from
linescan imagers: ``Spacecraft jitter'' is caused by oscillations on
the spacecraft due to the movement of other spacecraft hardware.  All
spacecraft wobble around to some degree but some, especially Mars
Global Surveyor, are particularly succeptible.

Jitter causes wave-like distortions along the track of the satellite
orbit in DEMs produced from linescan camera images.  This effect can
be very subtle or quite pronounced, so it is important to check you
data products carefully for any sign of this type of artifact. The
following examples will show the typical distortions created by this
problem.

Note that the science teams of HiRISE and LROC are actively working on
detecting and correctly modeling jitter in their respective SPICE
data. If they succeed in this, the distortions will still being in the
raw imagery, but the jitter will no longer produce ripple artifacts in
the DEMs produced using ours or other stereo reconstruction software.

\subsection{Ceraunius Tholus}

Ceraunius Tholus is a steep volcano that is part of the Uranius group
on Mars. It can be found at 23.96 N and 262.60 E. This DEM crosses the
volcano's caldera.

\subsubsection*{Screenshot}

\begin{figure}[h!]
\centering
  \subfigure[{\tt 3D Rendering}]{\includegraphics[width=3in]{images/examples/mocna/ceraunius_tholus_mocna.png}}
  \hfil
  \subfigure[{\tt KML Screenshot}]{\includegraphics[width=3in]{images/examples/mocna/ceraunius_tholus_mocna_ge.png}}
\caption{Example output for MOC-NA of Ceraunius Tholus. Notice the presence of severe washboarding artifacts due to spacecraft ``jitter.''}
\label{fig:mocna_ceraunius_example}
\end{figure}

\subsubsection*{Commands}

\begin{verbatim}
    % Download M0806047.img & R0701361.img
    moc2isis f=M0806047.img t=M0806047.cub mapping=no
    moc2isis f=R0701361.img t=R0701361.cub mapping=no
    cam2map from=M0806047.cub to=M0806047.map.cub
    cam2map from=R0701361.cub map=M0806047.map.cub to=R0701361.map.cub matchmap=true
    mkdir result
    stereo M0806047.map.cub R0701361.map.cub result/output
\end{verbatim}

\subsubsection*{Stereo Default}

\begin{verbatim}
    ### PREPROCESSING

    DO_INTERESTPOINT_ALIGNMENT 0
    INTERESTPOINT_ALIGNMENT_SUBSAMPLING 0
    DO_EPIPOLAR_ALIGNMENT 0

    FORCE_USE_ENTIRE_RANGE 1
    DO_INDIVIDUAL_NORMALIZATION 1

    PREPROCESSING_FILTER_MODE 2

    SLOG_KERNEL_WIDTH 1.5

    ### CORRELATION

    COST_BLUR 12
    COST_MODE 2

    H_KERNEL 25
    V_KERNEL 25

    H_CORR_MIN -12
    H_CORR_MAX 26
    V_CORR_MIN -50
    V_CORR_MAX 15

    SUBPIXEL_MODE 2

    SUBPIXEL_H_KERNEL 21
    SUBPIXEL_V_KERNEL 21

    ### FILTERING

    FILL_HOLES 1

    RM_H_HALF_KERN 5
    RM_V_HALF_KERN 5
    RM_MIN_MATCHES 60 # Units = percent
    RM_THRESHOLD 3
    RM_CLEANUP_PASSES 1

    ### DOTCLOUD

    NEAR_UNIVERSE_RADIUS 0.0
    FAR_UNIVERSE_RADIUS 0.0
\end{verbatim}

\subsection{North Tharsis}

The Malin Space Science System's website describes this image as the
`Throughs and terraces in northern Tharsis'. This DEM is located at
20.20 N and 118.18 W on Mars.

\subsubsection*{Screenshot}

\begin{figure}[h!]
\centering
  \subfigure[{\tt 3D Rendering}]{\includegraphics[width=3in]{images/examples/mocna/n_tharsis_mocna.png}}
  \hfil
  \subfigure[{\tt KML Screenshot}]{\includegraphics[width=3in]{images/examples/mocna/n_tharsis_mocna_ge.png}}
\caption{Example output for MOC-NA of North Tharsis.}
\label{fig:mocna_n_tharsis_example}
\end{figure}

\subsubsection*{Commands}

\begin{verbatim}
    % Download M0803097.img & S0701420.img
    moc2isis f=M0803097.img t=M0803097.cub mapping=no
    moc2isis f=S0701420.img t=S0701420.cub mapping=no
    cam2map from=M0803097.cub to=M0803097.map.cub
    cam2map from=S0701420.cub map=M0803097.map.cub to=S0701420.map.cub matchmap=true
    mkdr result
    stereo M0803097.map.cub S0701420.map.cub result/output
\end{verbatim}

\subsubsection*{Stereo Default}

\begin{verbatim}
    ### PREPROCESSING

    DO_INTERESTPOINT_ALIGNMENT 0
    INTERESTPOINT_ALIGNMENT_SUBSAMPLING 0
    DO_EPIPOLAR_ALIGNMENT 0

    FORCE_USE_ENTIRE_RANGE 1
    DO_INDIVIDUAL_NORMALIZATION 1

    PREPROCESSING_FILTER_MODE 2

    SLOG_KERNEL_WIDTH 1.5

    ### CORRELATION

    COST_BLUR 12
    COST_MODE 2

    H_KERNEL 25
    V_KERNEL 25

    H_CORR_MIN -50
    H_CORR_MAX 0
    V_CORR_MIN -85
    V_CORR_MAX 0

    SUBPIXEL_MODE 2

    SUBPIXEL_H_KERNEL 21
    SUBPIXEL_V_KERNEL 21

    ### FILTERING

    FILL_HOLES 1

    RM_H_HALF_KERN 5
    RM_V_HALF_KERN 5
    RM_MIN_MATCHES 60 # Units = percent
    RM_THRESHOLD 3
    RM_CLEANUP_PASSES 1

    ### DOTCLOUD

    NEAR_UNIVERSE_RADIUS 0.0
    FAR_UNIVERSE_RADIUS 0.0
\end{verbatim}


\section{Lunar Reconaissance Orbiter LROC-NA}

\subsection{Lee-Lincoln Scarp}

This stereo pair covers the Taurus-Littrow valley on the Moon where,
on December 11, 1972, the astronauts of Apollo 17 landed. However,
this stereo pair does not contain the landing site.  It is slightly
west; focusing on the Lee-Lincoln scarp that is on North Massif. The
scarp is an 80 m high feature that is the only visible sign of a deep
fault.

\subsubsection*{Screenshot}

\begin{figure}[h!]
\centering
  \subfigure[{\tt 3D Rendering}]{\includegraphics[width=3in]{images/examples/lrocna/lroc-na-example.png}}
  \hfil
  \subfigure[{\tt KML Screenshot}]{\includegraphics[width=3in]{images/examples/lrocna/lroc-na-ge_example.png}}
\caption{Example output possible with a LROC NA stereo pair, using only a single CCDs from observations.}
\label{fig:lroc-na-example}
\end{figure}

\subsubsection*{Commands}

\begin{verbatim}
    % Download nacl00002db8.* & nacl00004c86.*
    % process with unreleased tools to make cubes
    cam2map from=nacl00002db8.cub to=nacl00002db8.map.cub
    cam2map from=nacl00004c86.cub map=nacl00002db8.map.cub ...
            to=nacl00004c86.map.cub matchmap=true
    mkdir result
    stereo nacl00002db8.map.cub nacl00004c86.map.cub result/output
\end{verbatim}

\subsubsection*{Stereo Default}

\begin{verbatim}
    ### PREPROCESSING

    DO_INTERESTPOINT_ALIGNMENT 0
    INTERESTPOINT_ALIGNMENT_SUBSAMPLING 0
    DO_EPIPOLAR_ALIGNMENT 0

    FORCE_USE_ENTIRE_RANGE 1
    DO_INDIVIDUAL_NORMALIZATION 0

    PREPROCESSING_FILTER_MODE 2

    SLOG_KERNEL_WIDTH 1.5

    ### CORRELATION

    COST_BLUR 12
    COST_MODE 2

    H_KERNEL 29
    V_KERNEL 29

    H_CORR_MIN -425
    H_CORR_MAX 150
    V_CORR_MIN -100
    V_CORR_MAX 100

    SUBPIXEL_MODE 2

    SUBPIXEL_H_KERNEL 25
    SUBPIXEL_V_KERNEL 25

    ### FILTERING

    FILL_HOLES 1

    RM_H_HALF_KERN 5
    RM_V_HALF_KERN 5
    RM_MIN_MATCHES 60 # Units = percent
    RM_THRESHOLD 3
    RM_CLEANUP_PASSES 1

    ### DOTCLOUD

    NEAR_UNIVERSE_RADIUS 0.0
    FAR_UNIVERSE_RADIUS 0.0
\end{verbatim}


\section{Apollo 15 Metric Camera Images}

Apollo Metric images were all taken at regular intervals, which means
that the same stereo.default can be used for all sequential pairs of
images. Apollo Metric images are ideal for stereo processing.  They
produce consistent, excellent results.

The scans performed by ASU are sufficiently detailed to exhibit film
grain at the highest resolution.  The amount of noise at the full
resolution is not helpful for the correlator, so we recommended
subsampling the images by a factor of 4.

Currently the tools to ingest Apollo TIFFs into ISIS are not
available, but these images should soon be released into the PDS for
general public usage.

\subsection{Ansgarius C}

Ansgarius C is a small crater on the west edge of the farside of the
Moon near the equator. It is east of Kapteyn A and B.

\subsubsection*{Screenshot}

\begin{figure}[h!]
\centering
  \subfigure[{\tt 3D Rendering}]{\includegraphics[width=3in]{images/examples/metric/metric_example.png}}
  \hfil
  \subfigure[{\tt KML Screenshot}]{\includegraphics[width=3in]{images/examples/metric/metric_ge_example.png}}
\caption{Example output possible with Apollo Metric frames AS15-M-2380 and AS15-M-2381.}
\label{fig:metric_example}
\end{figure}

\subsubsection*{Commands}

\begin{verbatim}
    % Process tif files with not yet released commands %
    reduce from=AS15-M-2380.cub to=sub4-AS15-M-2380.cub sscale=4 lscale=4
    reduce from=AS15-M-2381.cub to=sub4-AS15-M-2381.cub sscale=4 lscale=4
    spiceinit from=sub4-AS15-M-2380.cub
    spiceinit from=sub4-AS15-M-2381.cub
    ipfind --max 10000 sub4*.cub
    ipmatch -i 50 -r homography sub4*.cub
    mkdir result
    stereo sub4-AS15-M-2380.cub sub4-AS15-M-2381.cub result/output
\end{verbatim}

\subsubsection*{Stereo Default}

\begin{verbatim}
    ### PREPROCESSING

    DO_INTERESTPOINT_ALIGNMENT 1
    INTERESTPOINT_ALIGNMENT_SUBSAMPLING 0
    DO_EPIPOLAR_ALIGNMENT 0

    FORCE_USE_ENTIRE_RANGE 1
    DO_INDIVIDUAL_NORMALIZATION 0

    PREPROCESSING_FILTER_MODE 3

    SLOG_KERNEL_WIDTH 1.5

    ### CORRELATION

    COST_MODE 2
    COST_BLUR 25

    H_KERNEL 35
    V_KERNEL 35

    H_CORR_MIN -250
    H_CORR_MAX 250
    V_CORR_MIN -70
    V_CORR_MAX 100

    SUBPIXEL_MODE 2

    SUBPIXEL_H_KERNEL 25
    SUBPIXEL_V_KERNEL 25

    # Hidden advanced function
    CORRSCORE_REJECTION_THRESHOLD 1.4

    ### FILTERING

    FILL_HOLES 1

    RM_H_HALF_KERN 5
    RM_V_HALF_KERN 5
    RM_MIN_MATCHES 60 # Units = percent
    RM_THRESHOLD 3
    RM_CLEANUP_PASSES 1

    ### DOTCLOUD

    NEAR_UNIVERSE_RADIUS 0.0
    FAR_UNIVERSE_RADIUS 0.0
\end{verbatim}


\section{MESSENGER MDIS}

These results are a proof of concept showing off the strength of
building the Stereo Pipeline on top of ISIS. Support for
processing MDIS stereo pairs was not a goal during our design of the
software, but the fact that an MDIS camera model exists in ISIS means
that it too can be processed by the Stereo Pipeline.

For future mappers, we suggest checking out Mercury Flyby 3 data which
was not available at the time of this writing. Flyby 3 and Flyby 2
seem to have covered some of the same terrain with the narrow angle
camera.

\subsection{Wide Angle on flyby 2}

In most flyby imagery it is very hard to find good stereo pairs. This
pair was taken from a single flyby just seconds apart. Note also that
this pair is taken from different wave lengths \emph{(The letter at
  the end of the file designates the current filter being used on the
  wide angle camera)}. Unfortunately there is not enough of a
perspective change here to make anything other than the spherical
surface, but that alone is still an interesting result nonetheless.

\subsubsection*{Screenshot}

\begin{figure}[h!]
  \begin{center}
  \includegraphics[width=5in]{images/examples/mdis/mdis_wide_example.png}
  \end{center}
  \caption{ A rough attempt at stereo reconstruction from MDIS imagery. }
  \label{fig:mdis_attempt}
\end{figure}

\subsubsection*{Commands}

\begin{verbatim}
    mdis2isis from=EW0108825359A.IMG to=EW0108825359A.cub
    mdis2isis from=EW0108825379C.IMG to=EW0108825379C.cub
    spiceinit from=EW0108825359A.cub
    spiceinit from=EW0108825359C.cub
    ipfind --max 10000 *.cub
    ipmatch -i 10 -r homography *.cub
    mkdir result
    stereo EW0108825359A.cub EW0108825379C.cub stereo/output
\end{verbatim}

\subsubsection*{Stereo Default}

\begin{verbatim}
    ### PREPROCESSING

    DO_INTERESTPOINT_ALIGNMENT 1
    INTERESTPOINT_ALIGNMENT_SUBSAMPLING 0
    DO_EPIPOLAR_ALIGNMENT 0

    FORCE_USE_ENTIRE_RANGE 0
    DO_INDIVIDUAL_NORMALIZATION 1

    PREPROCESSING_FILTER_MODE 2

    SLOG_KERNEL_WIDTH 1.5

    ### CORRELATION

    COST_BLUR 5
    COST_MODE 0

    H_KERNEL 25
    V_KERNEL 25

    H_CORR_MIN -10
    H_CORR_MAX 10
    V_CORR_MIN -2
    V_CORR_MAX 2

    SUBPIXEL_MODE 2

    SUBPIXEL_H_KERNEL 19
    SUBPIXEL_V_KERNEL 19

    ### FILTERING

    FILL_HOLES 1

    RM_H_HALF_KERN 5
    RM_V_HALF_KERN 5
    RM_MIN_MATCHES 60 # Units = percent
    RM_THRESHOLD 3
    RM_CLEANUP_PASSES 1

    ### DOTCLOUD

    NEAR_UNIVERSE_RADIUS 0.0
    FAR_UNIVERSE_RADIUS 0.0

\end{verbatim}

\section{Cassini ISS NAC}

This is a proof of concept showing the strength of building the Stereo
Pipeline on top of ISIS.  Support for processing ISS NAC stereo pairs
was not a goal during our design of the software, but the fact that a
camera model exists in ISIS means that it too can be processed by the
Stereo Pipeline.

Identifying stereo pairs from satellites that do not orbit their
target is a challenge. We have found that one usually has to settle
with images that are not ideal: different lighting, little perspective
change, and little or no stereo parallax. So far we have had little
succes with Cassini's difficult data, but nonetheless we provide this
example as a potential starting point.

\subsection{Rhea}

Rhea is the second largest moon of Saturn and is roughly 1/3rd the
size of our own Moon. This example show, at the top right of both
images,a gaint impact basin named Tirawa that is 220 miles across. The
bright white area south of Tirawa is ejecta from a new crater.  The
lack of texture in this area poses a challenge for our correlator. The
results are just barely useful: the Tirawa impact can barely be made
out in the 3D data while the new crater and ejecta become only noise.

\subsubsection*{Screenshot}

\begin{figure}[p]
\centering
  \subfigure[{\tt Original Left Image}]{\includegraphics[width=3in]{images/examples/cassini/cassini_rhea_L.png}}
  \hfil
  \subfigure[{\tt Original Right Image}]{\includegraphics[width=3in]{images/examples/cassini/cassini_rhea_R.png}}
  \\
  \subfigure[{\tt Map Projected Left}]{\includegraphics[width=3in]{images/examples/cassini/cassini_rhea_map.png}}
  \hfil
  \subfigure[{\tt 3D Rendering}]{\includegraphics[width=3in]{images/examples/cassini/cassini_rhea.png}}
\caption{Example output of what is possible with Cassini's ISS NAC}
\label{fig:cassini-exampe}
\end{figure}

\subsubsection*{Commands}

\begin{verbatim}
    % Download N1511700120_1.IMG and W1567133629_1.IMG
    ciss2isis f=N1511700120_1.IMG t=N1511700120_1.cub
    ciss2isis f=W1567133629_1.IMG t=W1567133629_1.cub
    cisscal from=N1511700120_1.cub to=N1511700120_1.lev1.cub
    cisscal from=W1567133629_1.cub to=W1567133629_1.lev1.cub
    fillgap from=W1567133629_1.lev1.cub to=W1567133629_1.fill.cub %Only one image
                                                                  %exhibits the problem
    cubenorm from=N1511700120_1.lev1.cub to=N1511700120_1.norm.cub
    cubenorm from=W1567133629_1.fill.cub to=W1567133629_1.norm.cub
    cam2map from=N1511700120_1.norm.cub to=N1511700120_1.map.cub
    cam2map from=W1567133629_1.norm.cub map=N1511700120_1.map.cub ...
            to=W1567133629_1.map.cub matchmap=true;
    ls *.map.cub > fromlist
    ls N*.map.cub > holdlist
    equalizer fromlist=fromlist holdlist=holdlist
    mkdir result
    stereo N1511700120_1.map.equ.cub W1567133629_1.map.equ.cub result/rhea
\end{verbatim}

\subsubsection*{Stereo Default}

\begin{verbatim}
    ### PREPROCESSING

    DO_INTERESTPOINT_ALIGNMENT 0
    INTERESTPOINT_ALIGNMENT_SUBSAMPLING 0
    DO_EPIPOLAR_ALIGNMENT 0

    FORCE_USE_ENTIRE_RANGE 1
    DO_INDIVIDUAL_NORMALIZATION 1

    PREPROCESSING_FILTER_MODE 2

    SLOG_KERNEL_WIDTH 1.5

    ### CORRELATION

    COST_MODE 2
    COST_BLUR 11

    H_KERNEL 25
    V_KERNEL 25

    H_CORR_MIN -55
    H_CORR_MAX -5
    V_CORR_MIN -2
    V_CORR_MAX 10

    SUBPIXEL_MODE 3 # Experimental Subpixel Mode

    SUBPIXEL_H_KERNEL 21
    SUBPIXEL_V_KERNEL 21

    ### FILTERING

    FILL_HOLES 1

    RM_H_HALF_KERN 5
    RM_V_HALF_KERN 5
    RM_MIN_MATCHES 60 # Units = percent
    RM_THRESHOLD 3
    RM_CLEANUP_PASSES 1

    ### DOTCLOUD

    NEAR_UNIVERSE_RADIUS 0.0
    FAR_UNIVERSE_RADIUS 0.0
\end{verbatim}
