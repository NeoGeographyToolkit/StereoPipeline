
\chapter*{Credits}

This open source version of the Ames Stereo Pipeline (ASP) was
developed by the Intelligent Robotics Group (IRG), in the Intelligent
Systems Division at NASA Ames Research Center in Moffett Field, CA. It
builds on over ten years of IRG experience developing surface
reconstruction tools for terrestrial robotic field tests and planetary
exploration. \\

{\bf Principal Investigator, NASA Ames Planetary Mapping and Modeling Team}
\begin {itemize} 
\item Michael J.~Broxton (NASA/Carnegie Mellon University)\\ {\tt
  michael.broxton@nasa.gov}\\
\end{itemize}

{\bf Lead Developer \& Stereo Pipeline Project Lead}
\begin {itemize} 
\item Zachary Moratto (NASA/Stinger-Ghaffarian Technologies)
\end{itemize}

{\bf Development Team}
\begin{itemize}
\item Dr.~Ross Beyer (NASA/SETI Institute)
\item Dr.~Ara Nefian (NASA/Carnegie Mellon University)
\item Matthew Hancher (NASA)
\item Kyle Husmann (NASA/Educational Associates Program)
\item Mike Lundy (NASA/Stinger-Ghaffarian Technologies)
\item Vinh To (NASA/Stinger-Ghaffarian Technologies)
\end{itemize}

{\bf Contributing Developer \& Former IRG Terrain Reconstruction Lead:}
\begin{itemize}
\item Dr.\ Laurence Edwards (NASA)
\end{itemize}

A number of student interns have made significant contributions to
this project over the years: Sasha Aravkin (Washington State
University), Kyle Hussman (California Polytechnic Institute), Patrick
Mihelich (Stanford University), Melissa Bunte (Arizona State
University), Matthew Faulkner (Massachusetts Institute of Technology),
Todd Templeton (UC Berkeley), Morgon Kanter (Bard College), Kerri
Cahoy (Stanford University), and Ian Saxton (UC San Diego).

The open source stereo pipeline leverages stereo image processing
work, past and present, led by Dr. Laurence Edwards, Eric Zbinden
(formerly NASA/QSS Inc.), Dr.~Michael Sims (NASA), and others in the
Intelligent Systems Division at NASA Ames Research Center. It has
benefited substantially from the contributions of Dr.~Keith Nishihara
(formerly NASA/Stanford), Randy Sargent (NASA/Carnegie Mellon
University), Dr.~Judd Bowman (formerly NASA/QSS Inc.), Clay Kunz
(formerly NASA/QSS Inc.), and Dr.~Matthew Deans (NASA).

\section*{Acknowledgements}

The initial adaptation of Ames' stereo surface reconstruction tools to
orbital imagers was a result of a NASA funded, industry led project to
develop automated Digital Elevation Model generation techniques for
the Mars Global Surveyor (MGS) mission. Our work with the project's
Principle Investigator, Dr.~Michael Malin of Malin Space Science
Systems (MSSS), and Co-Investigator, Dr.~Laurence Edwards of NASA
Ames, inspired the idea of making stereo surface reconstruction
technology available and accessible to a broader community.  We thank
Dr.~Malin and Dr.~Edwards for providing the initial impetus that in no
small way made this open source stereo pipeline possible, and we thank
Dr.~Michael Caplinger, Joe Fahle and others at MSSS for their help and
technical assistance.

We'd also like to thank our friends and collaborators Dr.~Randolph
Kirk, Dr.~Brent Archinal, Trent Hare, and Dr.~Mark Rosiek of the USGS
Astrogeology Branch in Flagstaff, AZ for their encouragement and
willingness to share their experience and expertise by answering many
of our technical questions.  We also thank them for their ongoing
support and efforts to help us evaluate our work.  Thanks also to the
USGS ISIS team, especially Jeff Anderson and Kris Becker, for their
help in integrating this version of the stereo pipeline with the USGS
ISIS software package.

Thanks go also to Dr.~Mark Robinson, Jacob Danton, Ernest
Bowman-Cisneros, Dr.~Sam Laurence, and Melissa Bunte at Arizona State
University for their help in adapting the Ames' stereo pipeline to
lunar data sets including the Apollo Metric Camera.

Finally, we thank Melissa Bunte, Dr.~Ara Nefian, and Dr.~Laurence
Edwards for their contributions to this documentation, and Dr.~Terry
Fong (IRG Group Lead) for his management and support of the open
source \& public software release process.

Portions of this software were developed with support from the
following NASA funding sources: the Mars Technology Program, the Mars
Critical Data Products Initiative, the Mars Reconnaissance Orbiter
mission, the Applied Information Systems Research program grant
\#06-AISRP06-0142, the Lunar Advanced Science and Exploration Research
(LASER) program grant \#07-LASER07-0148, and the ESMD Lunar Mapping and
Modeling Program (LMMP).
